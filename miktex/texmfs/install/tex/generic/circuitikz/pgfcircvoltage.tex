% Copyright 2018-2022 by Romano Giannetti
% Copyright 2015-2022 by Stefan Lindner
% Copyright 2013-2022 by Stefan Erhardt
% Copyright 2007-2022 by Massimo Redaelli
%
% This file may be distributed and/or modified
%
% 1. under the LaTeX Project Public License and/or
% 2. under the GNU Public License.
%
% See the files gpl-3.0_license.txt and lppl-1-3c_license.txt for more details.

%%%%%%%%%%%%%%%%%%%%%%%%%%%%%%%%%%%%%%%%%%
%%  Voltage management

%% styles%<<<
\ctikzset{bipole voltage style/.style={}}
\tikzset{bipole voltage style/.code={
        \ctikzset{bipole voltage style/.style={#1}}
}}
\tikzset{bipole voltage append style/.code={
        \ctikzset{bipole voltage style/.append style={#1}}
}}

\ctikzset{v^>/.style = {
        v={#1},
        \circuitikzbasekey/bipole/voltage/direction = forward,
        \circuitikzbasekey/bipole/voltage/position = above
    }
}

\ctikzset{v^</.style = {
        v={#1},
        \circuitikzbasekey/bipole/voltage/direction = backward,
        \circuitikzbasekey/bipole/voltage/position = above
    }
}

\ctikzset{v_>/.style = {
        v={#1},
        \circuitikzbasekey/bipole/voltage/direction = forward,
        \circuitikzbasekey/bipole/voltage/position = below
    }
}

\ctikzset{v_</.style = {
        v={#1},
        \circuitikzbasekey/bipole/voltage/direction = backward,
        \circuitikzbasekey/bipole/voltage/position = below
    }
}

\ctikzset{v_/.style = {v={#1}, \circuitikzbasekey/bipole/voltage/position = below} }
\ctikzset{v^/.style = {v={#1}, \circuitikzbasekey/bipole/voltage/position  = above} }
\ctikzset{v>/.style = {v={#1}, \circuitikzbasekey/bipole/voltage/direction = forward} }
\ctikzset{v</.style = {v={#1}, \circuitikzbasekey/bipole/voltage/direction = backward} }%

\newif\ifpgfcirc@do@v@symbols\pgfcirc@do@v@symbolstrue
\ctikzset{no v symbols/.code={\pgfcirc@do@v@symbolsfalse}}
\ctikzset{v symbols/.code={\pgfcirc@do@v@symbolstrue}}

% Default position varies whether the component is a voltage source
% or not
\ctikzset{v/.code = {
        \pgfcirc@has@vtrue
        \ifpgf@circuit@bipole@override@source@vif
            \pgf@circuit@bipole@isvoltagefalse
            \pgf@circuit@bipole@iscurrentfalse
        \fi
        \ifpgf@circuit@bipole@isvoltage
            \pgfkeys{\circuitikzbasekey/bipole/voltage/position=above,
            \circuitikzbasekey/bipole/voltage/direction=forward}
        \else
            \ifpgf@circ@oldvoltagedirection
                \pgfkeys{\circuitikzbasekey/bipole/voltage/position=below,
                \circuitikzbasekey/bipole/voltage/direction=backward}
            \else
                \pgfkeys{\circuitikzbasekey/bipole/voltage/position=below,
                \circuitikzbasekey/bipole/voltage/direction=forward}
            \fi
        \fi
        \ifpgf@circ@oldvoltagedirection
            \ifpgf@circuit@bipole@iscurrent\ifpgf@circ@fixbatteries
                \pgfkeys{\circuitikzbasekey/bipole/voltage/position=below,
                \circuitikzbasekey/bipole/voltage/direction=forward}
        \fi\fi
        \else
        \ifpgf@circuit@bipole@iscurrent
            \ifpgf@circuit@bipole@current@backward
                \pgfkeys{\circuitikzbasekey/bipole/voltage/position=below,
                \circuitikzbasekey/bipole/voltage/direction=forward}
            \else
                \pgfkeys{\circuitikzbasekey/bipole/voltage/position=below,
                \circuitikzbasekey/bipole/voltage/direction=backward}
            \fi\fi\fi
            \pgfkeys{/tikz/circuitikz/bipole/voltage/label/name={#1}}
            \ctikzsetvalof{bipole/voltage/label/unit}{}
            \ifpgf@circ@siunitx
                \pgf@circ@handleSI{#1}
                \ifpgf@circ@siunitx@res
                    \edef\pgf@temp{\pgf@circ@handleSI@val}
                    \pgfkeyslet{/tikz/circuitikz/bipole/voltage/label/name}{\pgf@temp}
                    \edef\pgf@temp{\pgf@circ@handleSI@unit}
                    \pgfkeyslet{/tikz/circuitikz/bipole/voltage/label/unit}{\pgf@temp}
                \else
            \fi
            \else
        \fi
    }
}
% %>>>

% american voltage font selection and symbol definition
% the default font command is {} --- nothing
\def\pgf@circ@avfont{\ctikzvalof{voltage/american font}}
%
% plus and minus symbols (default is $+$ and $-$, see pgfcirc.defines.tex)
%
\def\pgf@circ@avplus{\ctikzvalof{voltage/american plus}}
\def\pgf@circ@avminus{\ctikzvalof{voltage/american minus}}

%%
\def\setscaledRlenforclass{%
    \csname pgf@sh@ma@\ctikzvalof{bipole/name}\endcsname
    \ifdefined\ctikzclass
        \edef\pgf@temp{/tikz/circuitikz/\ctikzclass/scale}
        \pgfkeysifdefined{\pgf@temp}
            {\pgf@circ@scaled@Rlen=\ctikzvalof{\ctikzclass/scale}\pgf@circ@Rlen}
            {\pgf@circ@scaled@Rlen=\pgf@circ@Rlen}
    \else
        \pgf@circ@scaled@Rlen=\pgf@circ@Rlen
    \fi
}

%% Output routine for generic bipoles%<<<
% put this to true to see the voltage label coordinate anchors
\newif\ifpgf@circ@debugv\pgf@circ@debugvfalse

\def\pgf@circ@drawvoltagegeneric{
    \pgfextra{
        % \typeout{KIND:\ctikzvalof{bipole/kind}\space RLEN:\the\pgf@circ@Rlen\space SCALED:\the\pgf@circ@scaled@Rlen}
        \ifnum \ctikzvalof{mirror value}=-1
            \ifpgf@circuit@bipole@inverted
                \def\distfromline{\ctikzvalof{voltage/distance from line}\pgf@circ@scaled@Rlen}
            \else
                \def\distfromline{-\ctikzvalof{voltage/distance from line}\pgf@circ@scaled@Rlen}
            \fi
        \else
            \ifpgf@circuit@bipole@inverted
                    \def\distfromline{-\ctikzvalof{voltage/distance from line}\pgf@circ@scaled@Rlen}
                \else
                    \def\distfromline{\ctikzvalof{voltage/distance from line}\pgf@circ@scaled@Rlen}
            \fi
        \fi
        \ifpgf@circuit@bipole@voltage@below
            \def\pgf@circ@voltage@angle{90}
        \else
            \def\pgf@circ@voltage@angle{-90}
        \fi
        \edef\pgf@temp{/tikz/circuitikz/bipoles/\ctikzvalof{bipole/kind}/voltage/distance from node}
        \pgfkeysifdefined{\pgf@temp}
            { \edef\distancefromnode{\ctikzvalof{bipoles/\ctikzvalof{bipole/kind}/voltage/distance from node}} }
            { \edef\distancefromnode{\ctikzvalof{voltage/distance from node}} }
        \edef\pgf@temp{/tikz/circuitikz/bipoles/\ctikzvalof{bipole/kind}/voltage/bump b}
        \pgfkeysifdefined{\pgf@temp}
            { \edef\bumpb{\ctikzvalof{bipoles/\ctikzvalof{bipole/kind}/voltage/bump b}} }
            { \edef\bumpb{\ctikzvalof{voltage/bump b}} }
        \edef\shiftv{\ctikzvalof{voltage/shift}}
        % additional per-bipole voltage shift (internal)
        \edef\pgf@temp{/tikz/circuitikz/bipoles/\ctikzvalof{bipole/kind}/voltage/additional shift}
        \pgfkeysifdefined{\pgf@temp}
        {
            \edef\addvshift{\ctikzvalof{bipoles/\ctikzvalof{bipole/kind}/voltage/additional shift}}
        }
        {
            \edef\addvshift{0}
        }
        \newdimen{\absvshift}
        \pgfmathsetlength{\absvshift}{(1+\shiftv+\addvshift)*(\distfromline)}
        % reset anchor if american and open
        \ifpgf@circuit@europeanvoltage
        \else
            \ifx\@@kind\@@open
                \def\pgf@circ@bipole@voltage@label@anchor{center}
            \fi
        \fi
        \ifpgf@circuit@bipole@voltage@raised
            \def\pgf@circ@bipole@voltage@label@anchor{center}
            \pgfmathsetlength{\absvshift}{\absvshift+sign(\absvshift)*height{"Q"}} % with the current font.
        \fi
    }
    % %\pgf@circ@Rlen/\ctikzvalof{current arrow scale} is equal to the length of the currarrow
    %absolute move, minimum space is length of arrowhead
    coordinate (pgfcirc@midtmp) at ($(\tikztostart) ! \pgf@circ@Rlen/\ctikzvalof{current arrow scale} ! (pgfcirc@anchorstartnode)$)
    coordinate (pgfcirc@midtmp) at ($(pgfcirc@midtmp) ! \distancefromnode ! (pgfcirc@anchorstartnode)$)
    coordinate (pgfcirc@Vfrom@flat) at (pgfcirc@midtmp)
    %absolute move, minimum space is length of arrowhead
    coordinate (pgfcirc@midtmp) at ($(\tikztotarget) ! \pgf@circ@Rlen/\ctikzvalof{current arrow scale} ! (pgfcirc@anchorendnode)$)
    coordinate (pgfcirc@midtmp) at ($(pgfcirc@midtmp) ! \distancefromnode ! (pgfcirc@anchorendnode)$)
    coordinate (pgfcirc@Vto@flat) at (pgfcirc@midtmp)
    coordinate (pgfcirc@mid) at ($(pgfcirc@Vfrom@flat)!0.5!(pgfcirc@Vto@flat)$)

    \ifpgf@circuit@bipole@voltage@below
    % see comments for the "above" part (similar)
        \ifpgf@circuit@europeanvoltage
            \ifpgf@circuit@bipole@voltage@straight
                coordinate (pgfcirc@bottom) at (\ctikzvalof{bipole/name}.-90)
                coordinate (pgfcirc@Vto1) at ($(pgfcirc@mid)+(pgfcirc@bottom)-(pgfcirc@Vfrom@flat)$)
                coordinate (pgfcirc@Vfrom1) at ($(pgfcirc@mid)+(pgfcirc@bottom)-(pgfcirc@Vto@flat)$)
                coordinate (\pgfcirc@a@prefix-Vto)   at ($(pgfcirc@Vto1) ! \absvshift!90 :  (pgfcirc@Vfrom1)$)
                coordinate (\pgfcirc@a@prefix-Vfrom) at ($(pgfcirc@Vfrom1) ! \absvshift!-90 :  (pgfcirc@Vto1)$)
                coordinate (\pgfcirc@a@prefix-Vlab) at ($(\pgfcirc@a@prefix-Vto)!0.5!(\pgfcirc@a@prefix-Vfrom) $)
                coordinate (pgfcirc@Vdir) at (\pgfcirc@a@prefix-Vto)
            \else
                coordinate (\pgfcirc@a@prefix-Vto)   at ($(pgfcirc@Vto@flat) ! \absvshift!90 :  (pgfcirc@anchorendnode)$)
                coordinate (\pgfcirc@a@prefix-Vfrom) at ($(pgfcirc@Vfrom@flat) ! \absvshift!-90 :  (pgfcirc@anchorstartnode)$)
                coordinate (pgfcirc@Vcont1t) at ($(\ctikzvalof{bipole/name}.center) ! \bumpb ! (\ctikzvalof{bipole/name}.-110)$)
                coordinate (pgfcirc@Vcont2t) at ($(\ctikzvalof{bipole/name}.center) ! \bumpb ! (\ctikzvalof{bipole/name}.-70)$)
                coordinate (\pgfcirc@a@prefix-Vcont1) at ($(pgfcirc@Vcont1t) ! -\absvshift!90 : (pgfcirc@Vcont2t)$)
                coordinate (\pgfcirc@a@prefix-Vcont2) at ($(pgfcirc@Vcont2t) ! -\absvshift!-90 : (pgfcirc@Vcont1t)$)
                coordinate (\pgfcirc@a@prefix-Vlab) at ($(\pgfcirc@a@prefix-Vcont2)!0.5!(\pgfcirc@a@prefix-Vcont1)$)
                coordinate (pgfcirc@Vdir) at (\pgfcirc@a@prefix-Vcont2)
            \fi
        \else
            % we are in case of american here
            coordinate (\pgfcirc@a@prefix-Vto) at ($(pgfcirc@Vto@flat) ! \absvshift!90 :  (pgfcirc@anchorendnode)$)
            coordinate (\pgfcirc@a@prefix-Vfrom) at ($(pgfcirc@Vfrom@flat) ! \absvshift!-90 :  (pgfcirc@anchorstartnode)$)
            coordinate (pgfcirc@bottom) at (\ctikzvalof{bipole/name}.-90)
            coordinate (pgfcirc@Vdir0) at ($(pgfcirc@mid)+(pgfcirc@bottom)-(pgfcirc@Vfrom@flat)$)
            coordinate (\pgfcirc@a@prefix-Vlab) at ($(pgfcirc@bottom) !  \absvshift!-90 : (pgfcirc@Vdir0)$)
            coordinate (pgfcirc@Vdir) at ($(pgfcirc@mid)+(\pgfcirc@a@prefix-Vlab)-(pgfcirc@Vfrom@flat)$)
            \ifpgf@circuit@bipole@voltage@raised
                % move the from and to up to the level of Vlab
                coordinate (\pgfcirc@a@prefix-Vto) at ($(\pgfcirc@a@prefix-Vlab)+(pgfcirc@Vto@flat)-(pgfcirc@mid)$)
                coordinate (\pgfcirc@a@prefix-Vfrom) at ($(\pgfcirc@a@prefix-Vlab)+(pgfcirc@Vfrom@flat)-(pgfcirc@mid)$)
            \fi
        \fi
    \else
        \ifpgf@circuit@europeanvoltage
            \ifpgf@circuit@bipole@voltage@straight
                coordinate (pgfcirc@top) at (\ctikzvalof{bipole/name}.90)
                % move parallel to the component line at pgfcirc@top distance
                coordinate (pgfcirc@Vto1) at ($(pgfcirc@mid)+(pgfcirc@top)-(pgfcirc@Vfrom@flat)$)
                coordinate (pgfcirc@Vfrom1) at ($(pgfcirc@mid)+(pgfcirc@top)-(pgfcirc@Vto@flat)$)
                % add the extra distance
                coordinate (\pgfcirc@a@prefix-Vto)   at ($(pgfcirc@Vto1) ! \absvshift!-90 :  (pgfcirc@Vfrom1)$)
                coordinate (\pgfcirc@a@prefix-Vfrom) at ($(pgfcirc@Vfrom1) ! \absvshift!90 :  (pgfcirc@Vto1)$)
                coordinate (\pgfcirc@a@prefix-Vlab) at ($(\pgfcirc@a@prefix-Vto)!0.5!(\pgfcirc@a@prefix-Vfrom) $)
                % direction line to shift the label later
                coordinate (pgfcirc@Vdir) at (\pgfcirc@a@prefix-Vto)
            \else
                % european voltages here
                coordinate (\pgfcirc@a@prefix-Vto) at ($(pgfcirc@Vto@flat) ! -\absvshift!90 :  (pgfcirc@anchorendnode)$)
                coordinate (\pgfcirc@a@prefix-Vfrom) at ($(pgfcirc@Vfrom@flat) ! -\absvshift!-90 :  (pgfcirc@anchorstartnode)$)
                % identify the two control points for the "arc" of the voltage
                coordinate (pgfcirc@Vcont1t) at ($(\ctikzvalof{bipole/name}.center) ! \bumpb ! (\ctikzvalof{bipole/name}.110)$)
                coordinate (pgfcirc@Vcont2t) at ($(\ctikzvalof{bipole/name}.center) ! \bumpb ! (\ctikzvalof{bipole/name}.70)$)
                % and shift them a bit
                coordinate (\pgfcirc@a@prefix-Vcont1) at ($(pgfcirc@Vcont1t) ! \absvshift!90 : (pgfcirc@Vcont2t)$)
                coordinate (\pgfcirc@a@prefix-Vcont2) at ($(pgfcirc@Vcont2t) ! \absvshift!-90 : (pgfcirc@Vcont1t)$)
                coordinate (\pgfcirc@a@prefix-Vlab) at ($(\pgfcirc@a@prefix-Vcont2)!0.5!(\pgfcirc@a@prefix-Vcont1)$)
                % direction line to shift the label later
                coordinate (pgfcirc@Vdir) at (\pgfcirc@a@prefix-Vcont2)
            \fi
        \else
            % we are in case of american here
            coordinate (\pgfcirc@a@prefix-Vto) at ($(pgfcirc@Vto@flat) ! \absvshift!-90 :  (pgfcirc@anchorendnode)$)
            coordinate (\pgfcirc@a@prefix-Vfrom) at ($(pgfcirc@Vfrom@flat) ! \absvshift!90 :  (pgfcirc@anchorstartnode)$)
            coordinate (pgfcirc@top) at (\ctikzvalof{bipole/name}.90)
            % move parallel to the component line
            coordinate (pgfcirc@Vdir0) at ($(pgfcirc@mid)+(pgfcirc@top)-(pgfcirc@Vfrom@flat)$)
            % and add the extra distance
            coordinate (\pgfcirc@a@prefix-Vlab) at ($(pgfcirc@top) !  \absvshift!90 : (pgfcirc@Vdir0)$)
            coordinate (pgfcirc@Vdir) at ($(pgfcirc@mid)+(\pgfcirc@a@prefix-Vlab)-(pgfcirc@Vfrom@flat)$)
            \ifpgf@circuit@bipole@voltage@raised
                % move the from and to up to the level of Vlab
                coordinate (\pgfcirc@a@prefix-Vto) at ($(\pgfcirc@a@prefix-Vlab)+(pgfcirc@Vto@flat)-(pgfcirc@mid)$)
                coordinate (\pgfcirc@a@prefix-Vfrom) at ($(\pgfcirc@a@prefix-Vlab)+(pgfcirc@Vfrom@flat)-(pgfcirc@mid)$)
            \fi
        \fi
    \fi
    \ifx\@@kind\@@open
        coordinate (\pgfcirc@a@prefix-Vto) at (pgfcirc@Vto@flat)
        coordinate (\pgfcirc@a@prefix-Vfrom) at (pgfcirc@Vfrom@flat)
    \fi
    \ifpgf@circ@debugv
        node [ocirc, fill=red] at (pgfcirc@anchorstartnode) {}
        node [ocirc, fill=blue] at (pgfcirc@anchorendnode) {}
        node [ocirc, fill=green] at (\pgfcirc@a@prefix-Vto) {}
        node [ocirc, fill=yellow] at (\pgfcirc@a@prefix-Vfrom) {}
        node [odiamondpole, fill=green!50!black] at (pgfcirc@Vto@flat) {}
        node [odiamondpole, fill=orange] at (pgfcirc@Vfrom@flat) {}
        \ifpgf@circuit@europeanvoltage
            \ifpgf@circuit@bipole@voltage@straight
            \else
                node [osquarepole, fill=red] at (\pgfcirc@a@prefix-Vcont1) {}
                node [osquarepole, fill=blue] at (\pgfcirc@a@prefix-Vcont2) {}
            \fi
        \fi
    \fi

    %
    % Now we draw the voltage things (only if not empty --- in which case we have been
    % called just to set the anchors)
    %
    \pgf@circ@ifkeyempty{bipole/voltage/label/name}\else
    \ifpgfcirc@do@v@symbols
    \ifpgf@circuit@europeanvoltage
        \ifpgf@circuit@bipole@voltage@straight
            \ifpgf@circuit@bipole@voltage@backward
                (\pgfcirc@a@prefix-Vto) --(\pgfcirc@a@prefix-Vfrom) node[currarrow, sloped,  allow upside down, pos=1,anchor=tip] {}
            \else
                (\pgfcirc@a@prefix-Vfrom) --(\pgfcirc@a@prefix-Vto) node[currarrow, sloped,  allow upside down, pos=1,anchor=tip] {}
            \fi
        \else
            \ifpgf@circuit@bipole@voltage@backward
                (\pgfcirc@a@prefix-Vto) .. controls (\pgfcirc@a@prefix-Vcont2)  and (\pgfcirc@a@prefix-Vcont1) ..
                node[currarrow, sloped,  allow upside down, pos=1, anchor=tip] {}
                (\pgfcirc@a@prefix-Vfrom)
            \else
                (\pgfcirc@a@prefix-Vfrom) .. controls (\pgfcirc@a@prefix-Vcont1)  and (\pgfcirc@a@prefix-Vcont2) ..
                node[currarrow, sloped,  allow upside down, pos=1, anchor=tip] {}
                (\pgfcirc@a@prefix-Vto)
            \fi
        \fi
    \else % american
        \ifpgf@circuit@bipole@voltage@backward
            \ifpgf@circ@oldvoltagedirection
                (\pgfcirc@a@prefix-Vfrom) node[inner sep=0, node font=\pgf@circ@avfont,
                    anchor=\pgf@circ@bipole@voltage@label@anchor]{\pgf@circ@avplus}
                (\pgfcirc@a@prefix-Vto) node[inner sep=0, node font=\pgf@circ@avfont,
                    anchor=\pgf@circ@bipole@voltage@label@anchor]{\pgf@circ@avminus}
            \else
                (\pgfcirc@a@prefix-Vfrom) node[inner sep=0, node font=\pgf@circ@avfont,
                    anchor=\pgf@circ@bipole@voltage@label@anchor]{\pgf@circ@avminus}
                (\pgfcirc@a@prefix-Vto) node[inner sep=0, node font=\pgf@circ@avfont,
                    anchor=\pgf@circ@bipole@voltage@label@anchor]{\pgf@circ@avplus}
            \fi
            \else
            \ifpgf@circ@oldvoltagedirection
                (\pgfcirc@a@prefix-Vfrom) node[inner sep=0, node font=\pgf@circ@avfont,
                    anchor=\pgf@circ@bipole@voltage@label@anchor]{\pgf@circ@avminus}
                (\pgfcirc@a@prefix-Vto) node[inner sep=0, node font=\pgf@circ@avfont,
                    anchor=\pgf@circ@bipole@voltage@label@anchor]{\pgf@circ@avplus}
            \else
                (\pgfcirc@a@prefix-Vfrom) node[inner sep=0, node font=\pgf@circ@avfont,
                    anchor=\pgf@circ@bipole@voltage@label@anchor]{\pgf@circ@avplus}
                (\pgfcirc@a@prefix-Vto) node[inner sep=0, node font=\pgf@circ@avfont,
                    anchor=\pgf@circ@bipole@voltage@label@anchor]{\pgf@circ@avminus}
            \fi
        \fi
    \fi
    \fi % closing the ...symbol
    \fi % Closing the ...ifempty
}
% %>>>

%% Output routine for voltage sources%<<<
\def\pgf@circ@drawvoltagegenerator{
    % the following is affected indirectly by voltage/shift, you can move the arrow with voltage/bump a.
    % it's not perfect, but I can't find the way to do it correctly...
    \pgfextra{
        \edef\shiftv{\ctikzvalof{voltage/shift}}
        % distance along the 60-120 axis
        \edef\pgf@temp{/tikz/circuitikz/bipoles/\ctikzvalof{bipole/kind}/voltage/bump a}
        \pgfkeysifdefined{\pgf@temp}
        {
            \edef\bumpa{\ctikzvalof{bipoles/\ctikzvalof{bipole/kind}/voltage/bump a}}
        }
        {
            \edef\bumpa{\ctikzvalof{voltage/bump a}}
        }
        % additional per-bipole voltage shift (internal)
        \edef\pgf@temp{/tikz/circuitikz/bipoles/\ctikzvalof{bipole/kind}/voltage/additional shift}
        \pgfkeysifdefined{\pgf@temp}
        {
            \edef\addvshift{\ctikzvalof{bipoles/\ctikzvalof{bipole/kind}/voltage/additional shift}}
        }
        {
            \edef\addvshift{0}
        }
        \pgfmathsetmacro{\bumpaplus}{\bumpa + 0.5*\shiftv} % coefficient added "by feel". Sorry.
    }
    \ifpgf@circuit@bipole@voltage@below
        coordinate (pgfcirc@Vfrom0) at ($(\ctikzvalof{bipole/name}.center) ! \bumpaplus ! (\ctikzvalof{bipole/name}.-120)$)
        coordinate (pgfcirc@Vto0) at ($(\ctikzvalof{bipole/name}.center) ! \bumpaplus ! (\ctikzvalof{bipole/name}.-60)$)
        coordinate (\pgfcirc@a@prefix-Vfrom) at ($ (pgfcirc@Vfrom0) ! \addvshift! -90: (pgfcirc@Vto0) $)
        coordinate (\pgfcirc@a@prefix-Vto) at ($ (pgfcirc@Vto0) ! \addvshift! 90: (pgfcirc@Vfrom0) $)
    \else
        coordinate (pgfcirc@Vfrom0) at ($ (\ctikzvalof{bipole/name}.center) ! \bumpaplus ! (\ctikzvalof{bipole/name}.120)$)
        coordinate (pgfcirc@Vto0) at ($ (\ctikzvalof{bipole/name}.center) ! \bumpaplus ! (\ctikzvalof{bipole/name}.60)$)
        coordinate (\pgfcirc@a@prefix-Vfrom) at ($ (pgfcirc@Vfrom0) ! \addvshift! 90: (pgfcirc@Vto0) $)
        coordinate (\pgfcirc@a@prefix-Vto) at ($ (pgfcirc@Vto0) ! \addvshift! -90: (pgfcirc@Vfrom0) $)
    \fi
    coordinate (\pgfcirc@a@prefix-Vlab) at ($(\pgfcirc@a@prefix-Vto)!0.5!(\pgfcirc@a@prefix-Vfrom) $)
    coordinate (pgfcirc@Vdir) at (\pgfcirc@a@prefix-Vto)
    \pgf@circ@ifkeyempty{bipole/voltage/label/name}\else
    \ifpgfcirc@do@v@symbols
    \ifpgf@circuit@europeanvoltage
        \ifpgf@circuit@bipole@voltage@backward
            (\pgfcirc@a@prefix-Vto)  -- node[currarrow, sloped,  allow upside down, pos=1, anchor=tip] {} (\pgfcirc@a@prefix-Vfrom)
        \else
            (\pgfcirc@a@prefix-Vfrom)  -- node[currarrow, sloped,  allow upside down, pos=1, anchor=tip] {} (\pgfcirc@a@prefix-Vto)
        \fi
        \else% american voltage
        \ifpgf@circuit@bipole@voltageoutsideofsymbol
            % if it is a battery, must put + and -

            \ifpgf@circ@fixbatteries
                \ifpgf@circuit@bipole@voltage@backward
                    (\pgfcirc@a@prefix-Vfrom)  node[node font=\pgf@circ@avfont] {\pgf@circ@avplus}
                    (\pgfcirc@a@prefix-Vto) node[node font=\pgf@circ@avfont] {\pgf@circ@avminus}
                \else
                    (\pgfcirc@a@prefix-Vfrom)  node[node font=\pgf@circ@avfont] {\pgf@circ@avminus}
                    (\pgfcirc@a@prefix-Vto) node[node font=\pgf@circ@avfont] {\pgf@circ@avplus}
                \fi
                \else
                \ifpgf@circuit@bipole@voltage@backward
                    (\pgfcirc@a@prefix-Vfrom)  node[node font=\pgf@circ@avfont] {\pgf@circ@avminus}
                    (\pgfcirc@a@prefix-Vto) node[node font=\pgf@circ@avfont] {\pgf@circ@avplus}
                \else
                    (\pgfcirc@a@prefix-Vfrom)  node[node font=\pgf@circ@avfont] {\pgf@circ@avplus}
                    (\pgfcirc@a@prefix-Vto) node[node font=\pgf@circ@avfont] {\pgf@circ@avminus}
                \fi
            \fi
        \fi
    \fi
    \fi % closes ... ifsymbol
    \fi % closes ...ifempty
}
% %>>>

%% Output routine (main entry point) %<<<
%% this is the entry point
%%
%% locally used dimensions
\newdimen{\pgfcirc@labelshift}
\newif\ifpgfcirc@v@curved % helper if --- true only if voltages are curved
\def\pgf@circ@drawvoltage{% node name
    \pgfextra{%
        % set the helper if...curved
        \ifpgf@circuit@europeanvoltage
            \ifpgf@circuit@bipole@voltage@straight
                \pgfcirc@v@curvedfalse
            \else
                \pgfcirc@v@curvedtrue
            \fi
        \else
            \pgfcirc@v@curvedfalse
        \fi
        % \typeout{V routine called with prefix: \pgfcirc@a@prefix}
        % Label anchors WARNING: indentation is probably wrong
        \edef\pgfcircmathresult{\expandafter\pgf@circ@stripdecimals\pgf@circ@direction\pgf@nil}
        \ifnum\pgfcircmathresult >4 \ifnum\pgfcircmathresult <86
        \ifpgf@circuit@bipole@voltage@below
            \def\pgf@circ@bipole@voltage@label@anchor{north west}
        \else
            \def\pgf@circ@bipole@voltage@label@anchor{south east}
        \fi
        \fi\fi
        \ifnum\pgfcircmathresult >85 \ifnum\pgfcircmathresult <95
        \ifpgf@circuit@bipole@voltage@below
            \def\pgf@circ@bipole@voltage@label@anchor{west}
        \else
            \def\pgf@circ@bipole@voltage@label@anchor{east}
        \fi
        \fi\fi
        \ifnum\pgfcircmathresult >94 \ifnum\pgfcircmathresult <176
        \ifpgf@circuit@bipole@voltage@below
            \def\pgf@circ@bipole@voltage@label@anchor{south west}
        \else \def\pgf@circ@bipole@voltage@label@anchor{north east}
        \fi
        \fi\fi
        \ifnum\pgfcircmathresult >175 \ifnum\pgfcircmathresult <185
        \ifpgf@circuit@bipole@voltage@below
            \def\pgf@circ@bipole@voltage@label@anchor{south}
        \else\def\pgf@circ@bipole@voltage@label@anchor{north}
        \fi
        \fi\fi
        \ifnum\pgfcircmathresult >184 \ifnum\pgfcircmathresult <266
        \ifpgf@circuit@bipole@voltage@below
            \def\pgf@circ@bipole@voltage@label@anchor{south east}
        \else\def\pgf@circ@bipole@voltage@label@anchor{north west}
        \fi
        \fi\fi
        \ifnum\pgfcircmathresult >265 \ifnum\pgfcircmathresult <275
        \ifpgf@circuit@bipole@voltage@below
            \def\pgf@circ@bipole@voltage@label@anchor{east}
        \else \def\pgf@circ@bipole@voltage@label@anchor{west}
        \fi
        \fi\fi
        \ifnum\pgfcircmathresult >274 \ifnum\pgfcircmathresult <356
        \ifpgf@circuit@bipole@voltage@below
            \def\pgf@circ@bipole@voltage@label@anchor{north east}
        \else\def\pgf@circ@bipole@voltage@label@anchor{south west}
        \fi
        \fi\fi
        \ifnum\pgfcircmathresult >-1 \ifnum\pgfcircmathresult <5
        \ifpgf@circuit@bipole@voltage@below
            \def\pgf@circ@bipole@voltage@label@anchor{north}
        \else\def\pgf@circ@bipole@voltage@label@anchor{south}
        \fi
        \fi\fi
        \ifnum\pgfcircmathresult >355 \ifnum\pgfcircmathresult <361
        \ifpgf@circuit@bipole@voltage@below
            \def\pgf@circ@bipole@voltage@label@anchor{north}
        \else\def\pgf@circ@bipole@voltage@label@anchor{south}
        \fi
        \fi\fi

        % export anchor position (if not needed, is always the same macro)
        \expandafter\xdef\csname pgfcirc@\pgfcirc@a@prefix-Vlab-anchor\endcsname{\pgf@circ@bipole@voltage@label@anchor}
        % this must be set *before* changing for mirroring and inverting; in that case
        % the xscale/yscale parameters take it into account
        \ifpgf@circuit@bipole@voltage@below
            \def\pgf@circ@bipole@voltage@label@where{-90}
        \else
            \def\pgf@circ@bipole@voltage@label@where{90}
        \fi

        % magic to counteract the scale and yscale effects (there should be a better way...)
        \ifnum \ctikzvalof{mirror value}=-1
            \ifpgf@circuit@bipole@voltage@below
                \pgf@circuit@bipole@voltage@belowfalse
            \else
                \pgf@circuit@bipole@voltage@belowtrue
            \fi
        \fi

        \ifpgf@circuit@bipole@inverted
            \ifpgf@circuit@bipole@voltage@below
                \pgf@circuit@bipole@voltage@belowfalse
            \else
                \pgf@circuit@bipole@voltage@belowtrue
            \fi
        \fi

        % take into account scaling
        \setscaledRlenforclass
        % set the macro for detecting open
        \edef\@@kind{\ctikzvalof{bipole/kind}}\edef\@@open{open}
        % start voltage label adjustment
        \ifpgf@circuit@europeanvoltage
            \ifpgf@circuit@bipole@voltage@straight
                % check for straight
                \edef\pgf@temp{/tikz/circuitikz/bipoles/\ctikzvalof{bipole/kind}/voltage/straight label distance}
                \pgfkeysifdefined{\pgf@temp}{%
                    \edef\labeldist{\ctikzvalof{bipoles/\ctikzvalof{bipole/kind}/voltage/straight label distance}}%
                    % \typeout{ST:ADJUSTED\space for\space \ctikzvalof{bipole/kind} \space at \space \stdist}
                }{\edef\labeldist{\ctikzvalof{voltage/straight label distance}}}
                \ifpgf@circ@debugv\edef\whichtypeshift{STR}\fi
                % do not labelshift for legacy straight open; 1.4 makes the shift null
                \ifx\@@kind\@@open\ifpgf@adjust@open@voltage\else\edef\labeldist{1.4}\fi\fi
            \else
                % check for european
                \edef\pgf@temp{/tikz/circuitikz/bipoles/\ctikzvalof{bipole/kind}/voltage/european label distance}
                \pgfkeysifdefined{\pgf@temp}{%
                    \edef\labeldist{\ctikzvalof{bipoles/\ctikzvalof{bipole/kind}/voltage/european label distance}}%
                    % \typeout{EU:ADJUSTED\space for\space \ctikzvalof{bipole/kind} \space at \space \eudist}
                }{ \edef\labeldist{\ctikzvalof{voltage/european label distance}}}
                \ifpgf@circ@debugv\edef\whichtypeshift{EUR}\fi
            \fi
        \else
            % check for american
            \edef\pgf@temp{/tikz/circuitikz/bipoles/\ctikzvalof{bipole/kind}/voltage/american label distance}
            \pgfkeysifdefined{\pgf@temp}{%
                \ifpgf@circuit@bipole@voltage@raised
                    % do not apply the shift if we are using raised american style
                    \edef\labeldist{1.4}% default value
                \else
                    \edef\labeldist{\ctikzvalof{bipoles/\ctikzvalof{bipole/kind}/voltage/american label distance}}%
                \fi
                % \typeout{AL:ADJUSTED\space for\space \ctikzvalof{bipole/kind} \space at \space \aldist}
            }{\edef\labeldist{\ctikzvalof{voltage/american label distance}}}
            \ifpgf@circ@debugv\edef\whichtypeshift{AME}\fi
        \fi
        % find the height of the bipole or use a default value
        \edef\pgf@temp{/tikz/circuitikz/bipoles/\ctikzvalof{bipole/kind}/height}
        \pgfkeysifdefined{\pgf@temp}
            {\pgfmathsetmacro{\partheightf}{0.5*\ctikzvalof{bipoles/\ctikzvalof{bipole/kind}/height}}
             \edef\partheight{\partheightf\pgf@circ@scaled@Rlen}}
            {\edef\partheight{(.5\pgf@circ@scaled@Rlen)}} %fallback to fixed value
        \ifpgf@circuit@bipole@isvoltage
            \pgfmathsetlength{\pgfcirc@labelshift}{(\labeldist-1.2)*\partheight}
        \else
            \pgfmathsetlength{\pgfcirc@labelshift}{(\labeldist-1.4)*\partheight}
        \fi
        % the value for the european was by default 1.4
        \pgfsetcornersarced{\pgfpointorigin}% do not use rounded corners!
        % stop the detection of open if I do not want special treatment
        \ifpgf@adjust@open@voltage\else\edef\@@open{this-will-nEver-match}\fi
        % \typeout{KIND\space\@@kind}
    }%end pgfextra

    \ifpgf@circuit@bipole@isvoltage
        \pgf@circ@drawvoltagegenerator
        % add fake cont1 and cont2 anchors for export
        coordinate (\pgfcirc@a@prefix-Vcont1) at ($(\pgfcirc@a@prefix-Vfrom) !0.5! (\pgfcirc@a@prefix-Vto)$)
        coordinate (\pgfcirc@a@prefix-Vcont2) at ($(\pgfcirc@a@prefix-Vfrom) !0.5! (\pgfcirc@a@prefix-Vto)$)
    \else
        \pgf@circ@drawvoltagegeneric
    \fi
    % % debugging
    % \pgfextra{%
    %     \typeout{LABEL\space KIND:\@@kind\space EU:\the\pgfcirc@eushift\space AL:\the\pgfcirc@alshift\space
    %         DIRECTION:\pgf@circ@bipole@voltage@label@where}
    %     \pgf@circ@debugvtrue}

    % move a bit if requested
    coordinate (\pgfcirc@a@prefix-Vlab) at ($(\pgfcirc@a@prefix-Vlab) ! \pgfcirc@labelshift ! \pgf@circ@bipole@voltage@label@where :(pgfcirc@Vdir)$)

    % check for the case of american AND open
    \ifpgf@circuit@europeanvoltage
    \else
        \ifx\@@kind\@@open
        % override pgfcirc@Vlab
            coordinate (\pgfcirc@a@prefix-Vlab) at ($(pgfcirc@Vfrom@flat)!0.5!(pgfcirc@Vto@flat)$)
        \fi
    \fi

    \ifpgf@circ@debugv
            node [odiamondpole, color=blue] at (\pgfcirc@a@prefix-Vlab) {}
            node [odiamondpole, color=red] at (pgfcirc@Vdir) {}
            node [overlay, red, font=\tiny, anchor=south east, align=right] at(pgfcirc@Vdir)
            {\whichtypeshift:\the\pgfcirc@labelshift\\ DIR:\pgf@circ@bipole@voltage@label@where}
    \fi

    % put the node only if it's not empty
    \pgf@circ@ifkeyempty{bipole/voltage/label/name}\else
        node [anchor=\pgf@circ@bipole@voltage@label@anchor, inner sep=2pt,
        \circuitikzbasekey/bipole voltage style](\ctikzvalof{bipole/name}voltage)
        at (\pgfcirc@a@prefix-Vlab) {\pgf@circ@finallabels{voltage/label}}
    \fi

    \ifpgfcirc@v@curved\else
        % fake Vcont1 and Vcont2 for when they are exported --- in the middle of Vto and Vfrom
        coordinate (\pgfcirc@a@prefix-Vcont1) at ($(\pgfcirc@a@prefix-Vfrom) !0.5! (\pgfcirc@a@prefix-Vto)$)
        coordinate (\pgfcirc@a@prefix-Vcont2) at ($(\pgfcirc@a@prefix-Vfrom) !0.5! (\pgfcirc@a@prefix-Vto)$)
    \fi
    % revert from and to (and c1 - c2) if needed (simpler than rework the positioning above...)
    \ifpgf@circuit@bipole@voltage@backward
        \pgfcirc@swap@coordinates{\pgfcirc@a@prefix-Vfrom}{\pgfcirc@a@prefix-Vto}
        \pgfcirc@swap@coordinates{\pgfcirc@a@prefix-Vcont1}{\pgfcirc@a@prefix-Vcont2}
    \fi

}%end drawvoltages

% %>>>

%% these should be arguably moved to every component definition...

% special cases for voltage positions%<<<1
% the KIND is the node name without SHAPE
% See the definition above for meaning
% if bipoles/KIND/voltage/straight label distance is not defined, it uses the height
% if bipoles/KIND/voltage/additional shift is not defined, it is 0 (extra distance)
%
\ctikzset{bipoles/generic/voltage/distance from node/.initial=0.4}
\ctikzset{bipoles/generic/voltage/bump b/.initial=2}
%
\ctikzset{bipoles/ageneric/voltage/distance from node/.initial=.4}
\ctikzset{bipoles/ageneric/voltage/bump b/.initial=2}
%
\ctikzset{bipoles/fullgeneric/voltage/distance from node/.initial=.4}
\ctikzset{bipoles/fullgeneric/voltage/bump b/.initial=2}
%
\ctikzset{bipoles/memristor/voltage/distance from node/.initial=.4}
\ctikzset{bipoles/memristor/voltage/bump b/.initial=2}
%
\ctikzset{bipoles/tline/voltage/bump b/.initial=2.4}
%
\ctikzset{bipoles/varistor/voltage/bump b/.initial=2.4}
\ctikzset{bipoles/varistor/voltage/american label distance/.initial=1.8}
%
\ctikzset{bipoles/photoresistor/voltage/bump b/.initial=1.6}
%
\ctikzset{bipoles/thermistor/voltage/bump b/.initial=2.4}
\ctikzset{bipoles/thermistor/voltage/european label distance/.initial=0.8}
\ctikzset{bipoles/thermistorntc/voltage/bump b/.initial=1.6}
\ctikzset{bipoles/thermistorntc/voltage/european label distance/.initial=0.8}
\ctikzset{bipoles/thermistorptc/voltage/bump b/.initial=1.6}
\ctikzset{bipoles/thermistorptc/voltage/european label distance/.initial=0.8}
%
\ctikzset{bipoles/ccapacitor/voltage/bump b/.initial=2.2}
%
\ctikzset{bipoles/emptyzzdiode/voltage/bump b/.initial=2.5}
\ctikzset{bipoles/emptyzzdiode/voltage/european label distance/.initial=1.0}
\ctikzset{bipoles/fullzzdiode/voltage/bump b/.initial=2.5}
\ctikzset{bipoles/fullzzdiode/voltage/european label distance/.initial=1.0}
\ctikzset{bipoles/emptythyristor/voltage/bump b/.initial=2.0}
\ctikzset{bipoles/emptythyristor/voltage/european label distance/.initial=1.2}
\ctikzset{bipoles/fullthyristor/voltage/bump b/.initial=2.0}
\ctikzset{bipoles/fullthyristor/voltage/european label distance/.initial=1.2}
\ctikzset{bipoles/emptytriac/voltage/bump b/.initial=1.8}
\ctikzset{bipoles/emptytriac/voltage/european label distance/.initial=0.8}
\ctikzset{bipoles/fulltriac/voltage/bump b/.initial=1.8}
\ctikzset{bipoles/fulltriac/voltage/european label distance/.initial=0.8}
%
\ctikzset{bipoles/short/voltage/american label distance/.initial=2.8}
\ctikzset{bipoles/open/voltage/distance from node/.initial=0.3}
%
\ctikzset{bipoles/battery/voltage/bump a/.initial=1.4}
\ctikzset{bipoles/vsourceAM/voltage/american label distance/.initial=1.2}
\ctikzset{bipoles/cvsourceAM/voltage/american label distance/.initial=1.2}
%%>>>

\endinput
% vim: set fdm=marker fmr=%<<<,%>>>:
