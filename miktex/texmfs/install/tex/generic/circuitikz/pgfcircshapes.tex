% Copyright 2018-2022 by Romano Giannetti
% Copyright 2015-2022 by Stefan Lindner
% Copyright 2013-2022 by Stefan Erhardt
% Copyright 2007-2022 by Massimo Redaelli
%
% This file may be distributed and/or modified
%
% 1. under the LaTeX Project Public License and/or
% 2. under the GNU Public License.
%
% See the files gpl-3.0_license.txt and lppl-1-3c_license.txt for more details.
%
% This file has folding marks for vim (See last line).
%

%%%%%%%%%%%%%%%%%%%%%%%%%%%%%%%%%%%%%%%%
%%  Other shapes

%% Nothing: empty shape%<<<

\pgfdeclareshape{emptyshape}{
    \savedanchor{\northeast}{%
        \pgf@x=.5\wd\pgfnodeparttextbox%
        \pgf@y=.5\ht\pgfnodeparttextbox%
    }
    % geo anchors based on north-east
    \pgfcirc@northeast@symmetric@geoanchors
    \anchor{text}{\pgfpoint{-.5\wd\pgfnodeparttextbox}{\dimexpr.5\dp\pgfnodeparttextbox-.5\ht\pgfnodeparttextbox}}
    \anchor{center}{
        \pgfpointorigin
    }
}%
%>>>

%% Poles%<<<
%
% Provision for changing opacity. Only expert use, see the manual.
%
\ctikzset{poles/open fill opacity/.initial=1.0}% better not touch it
\tikzset{open poles opacity/.code={%
        \ctikzset{poles/open fill opacity=#1}%
}}
\ctikzset{poles/full fill opacity/.initial=1.0}% better not touch it
\tikzset{full poles opacity/.code={%
        \ctikzset{poles/full fill opacity=#1}%

}}

%
% Provision for changing default background
%

\ctikzset{open poles fill/.initial={white}}

%% Full terminal

\pgfdeclareshape{circ}{
    \anchor{center}{
        \pgfpointorigin
    }
    \savedanchor\northwest{%
        \pgf@y=\ctikzvalof{nodes width}\pgf@circ@Rlen
        \pgf@x=-\pgf@y
    }
    \anchor{center}{ \pgf@y=0pt \pgf@x=0pt }
    % geo anchors based on north-west
    \pgfcirc@northwest@symmetric@geoanchors
    \anchorborder{
        \pgf@circ@res@left=\pgf@x
        \pgf@circ@res@up=\pgf@y
        \pgfpointborderellipse{\pgfpoint{\pgf@circ@res@left}{\pgf@circ@res@up}
        }{\pgfpoint{\ctikzvalof{nodes width}*\pgf@circ@Rlen}{\ctikzvalof{nodes width}*\pgf@circ@Rlen}}
    }
    \pgf@circ@draw@component{
            \pgfpathcircle{\pgfpointorigin}{\ctikzvalof{nodes width}*\pgf@circ@Rlen}
            \pgf@circ@setcolor
            \pgf@circ@fill@strokecolor
            \pgfsetfillopacity{\ctikzvalof{poles/full fill opacity}}% normally 1.0
            \pgfusepath{draw,fill}
    }
}

%% Empty round terminal

\pgfdeclareshape{ocirc}{
    \anchor{center}{
        \pgfpointorigin
    }
    \savedanchor\northwest{%
        \pgf@y=\ctikzvalof{nodes width}\pgf@circ@Rlen
        \pgf@x=-\pgf@y
    }
    \anchor{center}{ \pgf@y=0pt \pgf@x=0pt }
    % geo anchors based on north-west
    \pgfcirc@northwest@symmetric@geoanchors
    \anchorborder{
        \pgf@circ@res@left=\pgf@x
        \pgf@circ@res@up=\pgf@y
        \pgfpointborderellipse{\pgfpoint{\pgf@circ@res@left}{\pgf@circ@res@up}
        }{\pgfpoint{\ctikzvalof{nodes width}*\pgf@circ@Rlen}{\ctikzvalof{nodes width}*\pgf@circ@Rlen}}
    }
    \pgf@circ@draw@component{
        \pgfpathcircle{\pgfpointorigin}{\ctikzvalof{nodes width}*\pgf@circ@Rlen}
        \pgf@circ@setcolor
        \pgfsetfillopacity{\ctikzvalof{poles/open fill opacity}}% normally 1.0
        \ifx\tikz@fillcolor\pgfutil@empty
            % set the default fill color to white
            \pgfsetfillcolor{\ctikzvalof{open poles fill}}
            % ...but override it if the class is defined!
            % note that this element has no class, but will inherit it when used
            % into another component
            \pgf@circ@setifdefinedfill{draw, fill}{draw, fill}
        \else
            \pgfsetfillcolor{\tikz@fillcolor}
        \fi
        \pgfusepath{draw,fill}
    }
}

%% Diamond terminal

\pgfdeclareshape{diamondpole}{
    \anchor{center}{
        \pgfpointorigin
    }
    \savedanchor\northwest{%
        \pgfmathsetlength{\pgf@y}{sqrt(2)*\ctikzvalof{nodes width}*\pgf@circ@Rlen}
        \pgf@x=-\pgf@y
    }
    \anchor{center}{ \pgf@y=0pt \pgf@x=0pt }
    % geo anchors based on north-west
    \pgfcirc@northwest@symmetric@geoanchors
    \anchorborder{
        % \typeout{IN\space X:\the\pgf@x\space Y:\the\pgf@y}
        \pgfmathsetmacro{\@@switchx}{ifthenelse(\pgf@x>0,1,-1)}
        \pgfmathsetmacro{\@@switchy}{ifthenelse(\pgf@y>0,1,-1)}
        \pgfmathsetlength{\pgf@xa}{abs(\pgf@x)}
        \pgfmathsetlength{\pgf@ya}{abs(\pgf@y)}
        \pgfextracty{\pgf@circ@res@up}{\northwest}
        % \typeout{MID\space X:\the\pgf@xa\space Y:\the\pgf@ya\space L:\the\pgf@circ@res@up}
        % \typeout{MID\space SX:\@@switchx\space SY:\@@switchy}
        \pgfpointintersectionoflines
            {\pgfpointorigin}{\pgfqpoint{\pgf@xa}{\pgf@ya}}
            {\pgfqpoint{0pt}{\pgf@circ@res@up}}{\pgfqpoint{\pgf@circ@res@up}{0pt}}
        % \typeout{CROSS \space X:\the\pgf@x\space Y:\the\pgf@y}
        \pgf@x=\@@switchx\pgf@x
        \pgf@y=\@@switchy\pgf@y
    }
    \pgf@circ@draw@component{
        \pgfmathsetlength{\pgf@circ@res@temp}{\ctikzvalof{nodes width}*\pgf@circ@Rlen}
        \pgftransformrotate{45}
        \pgfpathrectanglecorners
        {\pgfpoint{-\pgf@circ@res@temp}{-\pgf@circ@res@temp}}
        {\pgfpoint{\pgf@circ@res@temp}{\pgf@circ@res@temp}}
        \pgf@circ@setcolor
        \pgf@circ@fill@strokecolor
        \pgfsetfillopacity{\ctikzvalof{poles/full fill opacity}}% normally 1.0
        \pgfusepath{draw,fill}
    }
}

%% Diamond terminal, unfilled

\pgfdeclareshape{odiamondpole}{
    \anchor{center}{
        \pgfpointorigin
    }
    \savedanchor\northwest{%
        \pgfmathsetlength{\pgf@y}{sqrt(2)*\ctikzvalof{nodes width}*\pgf@circ@Rlen}
        \pgf@x=-\pgf@y
    }
    \anchor{center}{ \pgf@y=0pt \pgf@x=0pt }
    % geo anchors based on north-west
    \pgfcirc@northwest@symmetric@geoanchors
    \anchorborder{
        % \typeout{IN\space X:\the\pgf@x\space Y:\the\pgf@y}
        \pgfmathsetmacro{\@@switchx}{ifthenelse(\pgf@x>0,1,-1)}
        \pgfmathsetmacro{\@@switchy}{ifthenelse(\pgf@y>0,1,-1)}
        \pgfmathsetlength{\pgf@xa}{abs(\pgf@x)}
        \pgfmathsetlength{\pgf@ya}{abs(\pgf@y)}
        \pgfextracty{\pgf@circ@res@up}{\northwest}
        % \typeout{MID\space X:\the\pgf@xa\space Y:\the\pgf@ya\space L:\the\pgf@circ@res@up}
        % \typeout{MID\space SX:\@@switchx\space SY:\@@switchy}
        \pgfpointintersectionoflines
            {\pgfpointorigin}{\pgfqpoint{\pgf@xa}{\pgf@ya}}
            {\pgfqpoint{0pt}{\pgf@circ@res@up}}{\pgfqpoint{\pgf@circ@res@up}{0pt}}
        % \typeout{CROSS \space X:\the\pgf@x\space Y:\the\pgf@y}
        \pgf@x=\@@switchx\pgf@x
        \pgf@y=\@@switchy\pgf@y
    }
    \pgf@circ@draw@component{
        \pgfmathsetlength{\pgf@circ@res@temp}{\ctikzvalof{nodes width}*\pgf@circ@Rlen}
        \pgftransformrotate{45}
        \pgfpathrectanglecorners
        {\pgfpoint{-\pgf@circ@res@temp}{-\pgf@circ@res@temp}}
        {\pgfpoint{\pgf@circ@res@temp}{\pgf@circ@res@temp}}
        \pgf@circ@setcolor
        \pgfsetfillopacity{\ctikzvalof{poles/open fill opacity}}% normally 1.0
        \ifx\tikz@fillcolor\pgfutil@empty
            % set the default fill color to white
            \pgfsetfillcolor{\ctikzvalof{open poles fill}}
            % ...but override it if the class is defined!
            % note that this element has no class, but will inherit it when used
            % into another component
            \pgf@circ@setifdefinedfill{draw, fill}{draw, fill}
        \else
            \pgfsetfillcolor{\tikz@fillcolor}
        \fi
        \pgfusepath{draw,fill}
    }
}

%% square terminal, filled

\pgfdeclareshape{squarepole}{
    \anchor{center}{
        \pgfpointorigin
    }
    \savedanchor\northwest{%
        \pgfmathsetlength{\pgf@y}{\ctikzvalof{nodes width}*\pgf@circ@Rlen}
        \pgf@x=-\pgf@y
    }
    \anchor{center}{ \pgf@y=0pt \pgf@x=0pt }
    % geo anchors based on north-west
    \pgfcirc@northwest@symmetric@geoanchors
    \anchorborder{
        \pgf@xa=\pgf@x
        \pgf@ya=\pgf@y
        \pgfextracty{\pgf@circ@res@up}{\northwest}
        \pgfpointborderrectangle
            {\pgfqpoint{\pgf@xa}{\pgf@ya}}
            {\pgfqpoint{\pgf@circ@res@up}{\pgf@circ@res@up}}
    }
    \pgf@circ@draw@component{
        \pgfmathsetlength{\pgf@circ@res@temp}{\ctikzvalof{nodes width}*\pgf@circ@Rlen}
        \pgfpathrectanglecorners
        {\pgfpoint{-\pgf@circ@res@temp}{-\pgf@circ@res@temp}}
        {\pgfpoint{\pgf@circ@res@temp}{\pgf@circ@res@temp}}
        \pgf@circ@setcolor
        \pgf@circ@fill@strokecolor
        \pgfsetfillopacity{\ctikzvalof{poles/full fill opacity}}% normally 1.0
        \pgfusepath{draw,fill}
    }
}
%% square terminal, unfilled

\pgfdeclareshape{osquarepole}{
    \anchor{center}{
        \pgfpointorigin
    }
    \savedanchor\northwest{%
        \pgfmathsetlength{\pgf@y}{\ctikzvalof{nodes width}*\pgf@circ@Rlen}
        \pgf@x=-\pgf@y
    }
    \anchor{center}{ \pgf@y=0pt \pgf@x=0pt }
    % geo anchors based on north-west
    \pgfcirc@northwest@symmetric@geoanchors
    \anchorborder{
        \pgf@xa=\pgf@x
        \pgf@ya=\pgf@y
        \pgfextracty{\pgf@circ@res@up}{\northwest}
        \pgfpointborderrectangle
            {\pgfqpoint{\pgf@xa}{\pgf@ya}}
            {\pgfqpoint{\pgf@circ@res@up}{\pgf@circ@res@up}}
    }
    \pgf@circ@draw@component{
        \pgfmathsetlength{\pgf@circ@res@temp}{\ctikzvalof{nodes width}*\pgf@circ@Rlen}
        \pgfpathrectanglecorners
        {\pgfpoint{-\pgf@circ@res@temp}{-\pgf@circ@res@temp}}
        {\pgfpoint{\pgf@circ@res@temp}{\pgf@circ@res@temp}}
        \pgf@circ@setcolor
        \pgfsetfillopacity{\ctikzvalof{poles/open fill opacity}}% normally 1.0
        \ifx\tikz@fillcolor\pgfutil@empty
            % set the default fill color to white
            \pgfsetfillcolor{\ctikzvalof{open poles fill}}
            % ...but override it if the class is defined!
            % note that this element has no class, but will inherit it when used
            % into another component
            \pgf@circ@setifdefinedfill{draw, fill}{draw, fill}
        \else
            \pgfsetfillcolor{\tikz@fillcolor}
        \fi
        \pgfusepath{draw,fill}
    }
}

%% Fill for correct rectangular joins

\pgfdeclareshape{rectjoinfill}{
    \savedanchor{\northeast}{%
        \pgf@x=.5\pgflinewidth%
        \pgf@y=.5\pgflinewidth%
    }
    % geo anchors based on north-east
    \pgfcirc@northeast@symmetric@geoanchors
    \anchor{center}{
        \pgfpointorigin
    }
    \anchorborder{
        \pgf@circ@res@left=\pgf@x
        \pgf@circ@res@up=\pgf@y
    }
    \pgf@circ@draw@component{
        \pgfpathrectanglecorners
        {\pgfpoint{0}{.5\pgflinewidth}}
        {\pgfpoint{0}{-.5\pgflinewidth}}
        \pgf@circ@setcolor
        \pgf@circ@fill@strokecolor
        \pgfusepath{draw,fill}
    }
}
% %>>>

%% Arrows%<<<
%% transistor arrow

\def\pgf@circ@find@linescale{
    % find the scale inverse of the scale factor: line width do not scale
    % with scale=..., transform shape so we have to counteract it.
    \iftikz@fullytransformed % this is true if `transform shape` is active
        % from @Circumscribe https://tex.stackexchange.com/a/474035/38080
        % Note that this trick is not working inside a `spy` environment...
        \pgfgettransformentries{\scaleA}{\scaleB}{\scaleC}{\scaleD}{\whatevs}{\whatevs}%
        \pgfmathsetmacro{\@@factor}{1.0/sqrt(abs(\scaleA*\scaleD-\scaleB*\scaleC))}%
    \else
        \pgfmathsetmacro{\@@factor}{1.0}
    \fi
}

\pgfdeclareshape{trarrow}{%
    % this arrow is only filled but grows with the linewidth, more or less
    % like currarrow do
    \savedanchor{\northeast}{%
        \pgf@circ@res@step = \pgf@circ@Rlen
        \pgf@circ@find@linescale
        \divide \pgf@circ@res@step by \ctikzvalof{current arrow scale}
        \pgfpoint{0.7*\pgf@circ@res@step +0.5*\@@factor*\pgflinewidth}
            {0.8*\pgf@circ@res@step+0.7593*\@@factor*\pgflinewidth}
    }
    % The arrow size should be more or less the same of a currarrow, which is
    % both filled and stroke, for backward output compatibility (more or less)
    %
    %      angle \beta       W is \pgf@circ@Rlen/\ctikzvalof{current arrow scale}
    %    |-\__               currarrow as the tip at (W,0)
    %    |    |              and the upper tail at (-0.7*W, 0.8*W)
    %    |    \__            it then "overshoot" do to the linew width L
    %    |       \__ xangle \alpha
    %    ---0------->
    %
    %   \beta = atan(0.7/0.8)  \alpha=atan(0.8/1.7)
    %   tip overshoot is (L/2)/sin(\alpha) = 1.743*L only in x direction
    %   tail overshoot is -(L/2) in x, and (L/2)/sin(\beta) = 0.7539*L in y
    %
    \savedanchor{\northwest}{%
        \pgf@circ@res@step = \pgf@circ@Rlen
        \divide \pgf@circ@res@step by \ctikzvalof{current arrow scale}
        \pgf@circ@find@linescale
        \pgfpoint{-0.7*\pgf@circ@res@step -0.5*\@@factor*\pgflinewidth}
            {0.8*\pgf@circ@res@step+0.7593*\@@factor*\pgflinewidth}
    }
    \savedanchor{\tip}{%
        \pgf@circ@res@step = \pgf@circ@Rlen
        \divide \pgf@circ@res@step by \ctikzvalof{current arrow scale}
        \pgf@circ@find@linescale
        \pgfpoint{\pgf@circ@res@step + 1.743*\@@factor*\pgflinewidth}{0pt}
    }
    \anchor{north}{\northeast\pgf@x=0cm\relax}
    \anchor{east}{\northeast\pgf@y=0cm\relax}
    \anchor{south}{\northeast\pgf@y=-\pgf@y \pgf@x=0cm\relax}
    \anchor{west}{\northeast\pgf@y=0cm\pgf@x=-\pgf@x}
    \anchor{north east}{\northeast}
    \anchor{north west}{\northeast\pgf@x=-\pgf@x}
    \anchor{south east}{\northeast\pgf@y=-\pgf@y}
    \anchor{south west}{\northeast\pgf@y=-\pgf@y\pgf@x=-\pgf@x}
    \anchor{center}{
        \pgfpointorigin
    }
    \anchor{tip}{
        \tip
    }
    \anchor{btip}{% this anchor is behind the tip of half a linewidth
        \tip
        \pgf@circ@find@linescale
        \pgf@circ@res@temp=\@@factor\pgflinewidth
        \advance\pgf@x by -.5\pgf@circ@res@temp
    }
    \pgf@circ@draw@component{
        \northwest
        \pgf@circ@res@up=\pgf@y
        \pgf@circ@res@left=\pgf@x
        \tip
        \pgf@circ@res@step = \pgf@x
        %
        \pgfpathmoveto{\pgfpoint{\pgf@circ@res@left}{0pt}}
        \pgfpathlineto{\pgfpoint{\pgf@circ@res@left}{\pgf@circ@res@up}}
        \pgfpathlineto{\pgfpoint{1\pgf@circ@res@step}{0pt}}
        \pgfpathlineto{\pgfpoint{\pgf@circ@res@left}{-\pgf@circ@res@up}}
        \pgfpathclose
        \pgf@circ@fill@strokecolor
        \pgfusepath{fill} % just fill
    }
}

%% Current arrow

%% we need a phantom version of this shape for advanced v-i-f
%% use strange names to keep ot private
\newif\ifpgfcirc@really@draw@currarrow\pgfcirc@really@draw@currarrowtrue
\ctikzset{phantom@currarrow/.code=\pgfcirc@really@draw@currarrowfalse}
\ctikzset{normal@currarrow/.code=\pgfcirc@really@draw@currarrowtrue}

\pgfdeclareshape{currarrow}{
    \savedanchor{\northeast}{%
        \pgf@circ@res@step = \pgf@circ@Rlen
        \divide \pgf@circ@res@step by \ctikzvalof{current arrow scale}
        \pgf@x=.5\pgf@circ@res@step
        \pgf@y=\pgf@x%
    }
    \anchor{north}{\northeast\pgf@x=0cm\relax}
    \anchor{east}{\northeast\pgf@y=0cm\relax}
    \anchor{south}{\northeast\pgf@y=-\pgf@y \pgf@x=0cm\relax}
    \anchor{west}{\northeast\pgf@y=0cm\pgf@x=-\pgf@x}
    \anchor{north east}{\northeast}
    \anchor{north west}{\northeast\pgf@x=-\pgf@x}
    \anchor{south east}{\northeast\pgf@y=-\pgf@y}
    \anchor{south west}{\northeast\pgf@y=-\pgf@y\pgf@x=-\pgf@x}
    \anchor{center}{
        \pgfpointorigin
    }
    \anchor{tip}{
        \pgfpointorigin
        \pgf@circ@res@step = \pgf@circ@Rlen
        \divide \pgf@circ@res@step by \ctikzvalof{current arrow scale}
        \pgf@x	=\pgf@circ@res@step
    }
    \pgf@circ@draw@component{
        \ifpgfcirc@really@draw@currarrow
            \pgf@circ@reset@arrows@rounded
            \pgf@circ@res@step = \pgf@circ@Rlen
            \divide \pgf@circ@res@step by \ctikzvalof{current arrow scale}

            \pgfpathmoveto{\pgfpoint{-.7\pgf@circ@res@step}{0pt}}
            \pgfpathlineto{\pgfpoint{-.7\pgf@circ@res@step}{-.8\pgf@circ@res@step}}
            \pgfpathlineto{\pgfpoint{1\pgf@circ@res@step}{0pt}}
            \pgfpathlineto{\pgfpoint{-.7\pgf@circ@res@step}{.8\pgf@circ@res@step}}
            \pgfpathclose
            \pgf@circ@setcolor
            \pgf@circ@fill@strokecolor
            \pgfusepath{draw,fill}
        \fi
    }
}

%% Flow arrow
%% we need a phantom version of this shape for advanced v-i-f
%% use strange names to keep ot private
\newif\ifpgfcirc@really@draw@flowarrow\pgfcirc@really@draw@flowarrowtrue
\ctikzset{phantom@flowarrow/.code=\pgfcirc@really@draw@flowarrowfalse}
\ctikzset{normal@flowarrow/.code=\pgfcirc@really@draw@vlowarrowtrue}

\pgfdeclareshape{flowarrow}{
    \savedanchor{\northeast}{%
        \pgf@circ@res@step = \pgf@circ@Rlen
        \divide \pgf@circ@res@step by \ctikzvalof{current arrow scale}
        \pgf@y=.5\pgf@circ@res@step
        \pgf@circ@res@step = \pgf@circ@Rlen
        \divide \pgf@circ@res@step by 4
        \pgf@x=\pgf@circ@res@step%
    }
    \anchor{north}{\northeast\pgf@x=0cm\relax}
    \anchor{east}{\northeast\pgf@y=0cm\relax}
    \anchor{south}{\northeast\pgf@y=-\pgf@y \pgf@x=0cm\relax}
    \anchor{west}{\northeast\pgf@y=0cm\pgf@x=-\pgf@x}
    \anchor{north east}{\northeast}
    \anchor{north west}{\northeast\pgf@x=-\pgf@x}
    \anchor{south east}{\northeast\pgf@y=-\pgf@y}
    \anchor{south west}{\northeast\pgf@y=-\pgf@y\pgf@x=-\pgf@x}
    \anchor{text}{% text centered above
        \pgfpointorigin
        \pgfpoint{-.5\wd\pgfnodeparttextbox}{\dimexpr.5\dp\pgfnodeparttextbox+.5\ht\pgfnodeparttextbox}
    }
    \anchor{center}{
        \pgfpointorigin
    }
    \anchor{tip}{
        \pgfpointorigin
        \pgf@circ@res@step = \pgf@circ@Rlen
        \divide \pgf@circ@res@step by \ctikzvalof{current arrow scale}
        \pgf@x	=\pgf@circ@res@step
    }
    \pgf@circ@draw@component{
        \ifpgfcirc@really@draw@flowarrow
            \pgf@circ@reset@arrows@rounded
            \pgf@circ@res@step = \pgf@circ@Rlen
            \divide \pgf@circ@res@step by 4
            \pgfpathmoveto{\pgfpoint{-\pgf@circ@res@step}{0pt}}
            \pgfpathlineto{\pgfpoint{\pgf@circ@res@step}{0pt}}
            \pgf@circ@setcolor
            \pgfusepath{draw}
            \pgf@circ@fill@strokecolor
            \pgftransformshift{\pgfpoint{\pgf@circ@res@step}{0pt}}
            \pgfnode{currarrow}{tip}{}{}{\pgfusepath{fill}}
        \fi
    }
}

%% Input arrow

\pgfdeclareshape{inputarrow}{
    \savedanchor{\northeast}{% this is really not northeast, really -northwest
        \pgf@circ@res@step = \pgf@circ@Rlen
        \divide \pgf@circ@res@step by \ctikzvalof{current arrow scale}
        \pgf@y=.5\pgf@circ@res@step
        \pgf@x=1.7\pgf@circ@res@step
    }
    \anchor{north}{\northeast\pgf@x=0cm\relax}
    \anchor{east}{\northeast\pgf@y=0cm\relax\pgf@x=0cm\relax}
    \anchor{south}{\northeast\pgf@y=-\pgf@y \pgf@x=0cm\relax}
    \anchor{west}{\northeast\pgf@y=0cm\pgf@x=-\pgf@x}
    \anchor{north east}{\northeast\pgf@x=0cm\relax}
    \anchor{north west}{\northeast\pgf@x=-\pgf@x}
    \anchor{south east}{\northeast\pgf@y=-\pgf@y\pgf@x=0cm\relax}
    \anchor{south west}{\northeast\pgf@y=-\pgf@y\pgf@x=-\pgf@x}
    \savedanchor{\tip}{
        \pgfpointorigin
    }
    \anchor{center}{
        \tip
    }
    \anchor{tip}{
        \tip
    }
    \pgf@circ@draw@component{
        \pgf@circ@reset@arrows@rounded
        \pgf@circ@res@step = \pgf@circ@Rlen
        \divide \pgf@circ@res@step by 16
        \pgfpathmoveto{\pgfpoint{-1.7\pgf@circ@res@step}{0pt}}
        \pgfpathlineto{\pgfpoint{-1.7\pgf@circ@res@step}{-.8\pgf@circ@res@step}}
        \pgfpathlineto{\pgfpoint{0pt}{0pt}}
        \pgfpathlineto{\pgfpoint{-1.7\pgf@circ@res@step}{.8\pgf@circ@res@step}}
        \pgfpathclose
        \pgf@circ@setcolor
        \pgf@circ@fill@strokecolor
        \pgfusepath{fill}
    }
}
% %>>>

%% boxes%<<<
%% box

\pgfdeclareshape{box}{
    \anchor{center}{
        \pgfpointorigin
    }
    \pgf@circ@draw@component{
        \pgf@circ@res@step = \ctikzvalof{bipoles/twoport/width}\pgf@circ@Rlen
        \pgf@circ@res@step = 0.5\pgf@circ@res@step
        \pgf@circ@setlinewidth{bipoles}{\pgfstartlinewidth}
        \pgfpathrectanglecorners{\pgfpoint{-\pgf@circ@res@step}{\pgf@circ@res@step}}{\pgfpoint{\pgf@circ@res@step}{-\pgf@circ@res@step}}
        \pgf@circ@draworfill
    }
}

%% box scaled with blocks

\pgfdeclareshape{blockbox}{
    \saveddimen{\scaledRlen}{\pgfmathsetlength{\pgf@x}{\ctikzvalof{blocks/scale}\pgf@circ@Rlen}}
    \anchor{center}{
        \pgfpointorigin
    }
    \pgf@circ@draw@component{
        \pgfmathsetlength{\pgf@circ@scaled@Rlen}{\ctikzvalof{blocks/scale}\pgf@circ@Rlen}
        \pgf@circ@res@step = \ctikzvalof{bipoles/twoport/width}\pgf@circ@scaled@Rlen
        \pgf@circ@res@step = 0.5\pgf@circ@res@step
        \pgf@circ@setlinewidth{bipoles}{\pgfstartlinewidth}
        \pgfpathrectanglecorners{\pgfpoint{-\pgf@circ@res@step}{\pgf@circ@res@step}}{\pgfpoint{\pgf@circ@res@step}{-\pgf@circ@res@step}}
        \pgf@circ@draworfill
    }
}
% %>>>

%% crossings%<<<
% full nodes for wire crossing

\pgfdeclareshape{jump crossing}
{
    \savedanchor\northwest{%
        \pgf@y=\ctikzvalof{bipoles/crossing/size}\pgf@circ@Rlen
        \pgf@y=.5\pgf@y
        \pgf@x=-\pgf@y
    }
    \anchor{center}{ \pgf@y=0pt \pgf@x=0pt }
    \anchor{east}{ \northwest \pgf@y=0pt \pgf@x=-\pgf@x  }
    \anchor{e}{ \northwest \pgf@y=0pt \pgf@x=-\pgf@x  }
    \anchor{west}{ \northwest \pgf@y=0pt }
    \anchor{w}{ \northwest \pgf@y=0pt }
    \anchor{south}{ \northwest \pgf@x=0pt \pgf@y=-\pgf@y }
    \anchor{s}{ \northwest \pgf@x=0pt \pgf@y=-\pgf@y }
    \anchor{north}{ \northwest \pgf@x=0pt }
    \anchor{n}{ \northwest \pgf@x=0pt }
    \anchor{south west}{ \northwest \pgf@y=-\pgf@y }
    \anchor{north east}{ \northwest \pgf@x=-\pgf@x }
    \anchor{north west}{ \northwest }
    \anchor{south east}{ \northwest \pgf@x=-\pgf@x \pgf@y=-\pgf@y }
    \pgf@circ@draw@component{
        \northwest
        \pgf@circ@res@up = \pgf@y
        \pgf@circ@res@down = -\pgf@y
        \pgf@circ@res@right = -\pgf@x
        \pgf@circ@res@left = \pgf@x
        % horizontal jumper
        \pgfpathmoveto{\pgfpoint{\pgf@circ@res@left}{0pt}}
        \pgfpathlineto{\pgfpoint{0.4\pgf@circ@res@left}{0pt}}
        \pgfpatharc{0}{-180}{0.4*\pgf@circ@res@left}
        \pgfsetbeveljoin
        \pgfpathlineto{\pgfpoint{\pgf@circ@res@right}{0pt}}
        % vertical, broken path
        \pgfpathmoveto{\pgfpoint{0pt}{\pgf@circ@res@up}}
        \pgfpathlineto{\pgfpoint{0pt}{0.5\pgf@circ@res@up}}
        \pgfpathmoveto{\pgfpoint{0pt}{0.3\pgf@circ@res@up}}
        \pgfpathlineto{\pgfpoint{0pt}{\pgf@circ@res@down}}
        \pgfusepath{draw}

    }
}
\pgfdeclareshape{plain crossing}
{
    \savedanchor\northwest{%
        \pgf@y=\ctikzvalof{bipoles/crossing/size}\pgf@circ@Rlen
        \pgf@y=.5\pgf@y
        \pgf@x=-\pgf@y
    }
    \anchor{center}{ \pgf@y=0pt \pgf@x=0pt }
    \anchor{east}{ \northwest \pgf@y=0pt \pgf@x=-\pgf@x  }
    \anchor{e}{ \northwest \pgf@y=0pt \pgf@x=-\pgf@x  }
    \anchor{west}{ \northwest \pgf@y=0pt }
    \anchor{w}{ \northwest \pgf@y=0pt }
    \anchor{south}{ \northwest \pgf@x=0pt \pgf@y=-\pgf@y }
    \anchor{s}{ \northwest \pgf@x=0pt \pgf@y=-\pgf@y }
    \anchor{north}{ \northwest \pgf@x=0pt }
    \anchor{n}{ \northwest \pgf@x=0pt }
    \anchor{south west}{ \northwest \pgf@y=-\pgf@y }
    \anchor{north east}{ \northwest \pgf@x=-\pgf@x }
    \anchor{north west}{ \northwest }
    \anchor{south east}{ \northwest \pgf@x=-\pgf@x \pgf@y=-\pgf@y }
    \pgf@circ@draw@component{
        \northwest
        \pgf@circ@res@up = \pgf@y
        \pgf@circ@res@down = -\pgf@y
        \pgf@circ@res@right = -\pgf@x
        \pgf@circ@res@left = \pgf@x
        % horizontal jumper
        \pgfpathmoveto{\pgfpoint{\pgf@circ@res@left}{0pt}}
        \pgfpathlineto{\pgfpoint{\pgf@circ@res@right}{0pt}}
        % vertical, broken path
        \pgfpathmoveto{\pgfpoint{0pt}{\pgf@circ@res@up}}
        \pgfpathlineto{\pgfpoint{0pt}{0.1\pgf@circ@res@up}}
        \pgfpathmoveto{\pgfpoint{0pt}{0.1\pgf@circ@res@down}}
        \pgfpathlineto{\pgfpoint{0pt}{\pgf@circ@res@down}}
        \pgfusepath{draw}

    }
}
% %>>>

%% Connectors (BNC and IEC connectors; see https://github.com/circuitikz/circuitikz/issues/611)%<<<

% define new class
\ctikzset{connectors/scale/.initial=1.0}
\ctikzset{connectors/fill/.initial=none}
\ctikzset{connectors/thickness/.initial=none}
% parameters. To have round sockets, 3*height==2*width
\ctikzset{bipoles/iecconn/height/.initial=.2}
\ctikzset{bipoles/iecconn/width/.initial=.3}
% objects
\pgfcircdeclarebipolescaled{connectors}
{
    \anchor{plug center}{\northeast\pgf@y=0pt\divide\pgf@x by 3 }
    \anchor{socket center}{\northeast\pgf@y=0pt\pgf@x=-0.333333\pgf@x}
    % put the node text above and centered
    \anchor{text}{\pgfextracty{\pgf@circ@res@up}{\northeast}
        \pgfpoint{-.5\wd\pgfnodeparttextbox}{
            \dimexpr.5\ht\pgfnodeparttextbox+\pgf@circ@res@up\relax
        }
    }
}
{\ctikzvalof{bipoles/iecconn/height}}%symmetrical
{iecconn}
{\ctikzvalof{bipoles/iecconn/height}}
{\ctikzvalof{bipoles/iecconn/width}}
{
    \pgf@circ@setlinewidth{bipoles}{\pgfstartlinewidth}
    \pgfpathmoveto{\pgfpoint{\pgf@circ@res@left/3}{\pgf@circ@res@up}}
    \pgfpatharc{90}{-90}{0.66666\pgf@circ@res@left and \pgf@circ@res@up}
    \pgfusepath{draw}
    \pgfpathrectanglecorners{\pgfpoint{\pgf@circ@res@left/3}{\pgf@circ@res@up/2}}{\pgfpoint{\pgf@circ@res@right}{\pgf@circ@res@down/2}}
    \pgf@circ@fill@strokecolor
    \pgfusepath{draw, fill}
}
\pgfcirc@activate@bipole@simple{l}{iecconn}
\pgfcirc@style@to@style{iecconn}{iec connector}

\long\def\pgfcirc@declare@iecsocket#1#2#3{% #1 name, #2 anchors, #3 drawing code
    \pgfdeclareshape{#1}{%
        \savedmacro{\ctikzclass}{\edef\ctikzclass{connectors}}
        \saveddimen{\scaledRlen}{\pgfmathsetlength{\pgf@x}{\ctikzvalof{\ctikzclass/scale}\pgf@circ@Rlen}}
        \savedanchor\northwest{%
            \pgfmathsetlength{\pgf@circ@scaled@Rlen}{\ctikzvalof{\ctikzclass/scale}\pgf@circ@Rlen}
            \pgf@y=\ctikzvalof{bipoles/iecconn/height}\pgf@circ@scaled@Rlen
            \pgf@y=.5\pgf@y
            \pgf@x=-\ctikzvalof{bipoles/iecconn/width}\pgf@circ@scaled@Rlen
            \divide\pgf@x by 6
        }
        \pgfcirc@northwest@symmetric@geoanchors
        #2%
        \pgf@circ@draw@component{%
            \pgf@circ@scaled@Rlen=\scaledRlen
            \pgfstartlinewidth=\pgflinewidth
            \northwest
            \pgf@circ@res@up=\pgf@y
            \pgf@circ@res@left=\pgf@x
            \pgf@circ@setlinewidth{bipoles}{\pgfstartlinewidth}
            #3%
            \pgfusepath{draw}
        }
    }
}

\long\def\pgfcirc@declare@iecplug#1#2{% #1 name, #2 anchors (drawing code is the same)
    \pgfdeclareshape{#1}{%
        \savedmacro{\ctikzclass}{\edef\ctikzclass{connectors}}
        \saveddimen{\scaledRlen}{\pgfmathsetlength{\pgf@x}{\ctikzvalof{\ctikzclass/scale}\pgf@circ@Rlen}}
        \savedanchor\northwest{%
            \pgfmathsetlength{\pgf@circ@scaled@Rlen}{\ctikzvalof{\ctikzclass/scale}\pgf@circ@Rlen}
            \pgf@y=\ctikzvalof{bipoles/iecconn/height}\pgf@circ@scaled@Rlen
            \pgf@y=.25\pgf@y
            \pgf@x=-\ctikzvalof{bipoles/iecconn/width}\pgf@circ@scaled@Rlen
            \divide\pgf@x by 3
        }
        \pgfcirc@northwest@symmetric@geoanchors
        \anchor{plug center}{\pgfpointorigin}
        #2%
        \pgf@circ@draw@component{%
            \pgf@circ@scaled@Rlen=\scaledRlen
            \pgfstartlinewidth=\pgflinewidth
            \northwest
            \pgf@circ@res@up=\pgf@y
            \pgf@circ@res@left=\pgf@x
            \pgf@circ@setlinewidth{bipoles}{\pgfstartlinewidth}
            \pgfpathrectanglecorners{\pgfpoint{\pgf@circ@res@left}{\pgf@circ@res@up}}{\pgfpoint{-\pgf@circ@res@left}{-\pgf@circ@res@up}}
            \pgf@circ@setcolor
            \pgf@circ@fill@strokecolor
            \pgfusepath{draw, fill}
        }
    }
}

\pgfcirc@declare@iecsocket{iecsocketR}{%
    % notice: center is on the left side
    \anchor{center}{\northwest\pgf@y=0pt}
    \anchor{socket center}{\northwest\pgf@y=0pt}
    % put the node text above and to the left, ignore depth
    \anchor{text}{%
        \pgfextractx{\pgf@circ@res@left}{\northwest}
        \pgfextracty{\pgf@circ@res@up}{\northwest}
        \pgfpoint{-\pgf@circ@res@left}{%
            .5\ht\pgfnodeparttextbox+\pgf@circ@res@up
        }%
    }%
}{% drawing
    \pgfpathmoveto{\pgfpoint{\pgf@circ@res@left}{\pgf@circ@res@up}}
    \pgfpatharc{90}{270}{2\pgf@circ@res@left and \pgf@circ@res@up}
}
\pgfcirc@declare@iecsocket{iecsocketL}{%
    % notice: center is on the left side
    \anchor{center}{\northwest\pgf@y=0pt}
    \anchor{socket center}{\northwest\pgf@y=0pt\pgf@x=-\pgf@x}
    % put the node text above and to the left, ignore depth
    \anchor{text}{%
        \pgfextractx{\pgf@circ@res@left}{\northwest}
        \pgfextracty{\pgf@circ@res@up}{\northwest}
        \pgfpoint{-\wd\pgfnodeparttextbox+\pgf@circ@res@left}{%
            .5\ht\pgfnodeparttextbox+\pgf@circ@res@up
        }%
    }%
}{% drawing
    \pgfpathmoveto{\pgfpoint{-\pgf@circ@res@left}{\pgf@circ@res@up}}
    \pgfpatharc{90}{-90}{2\pgf@circ@res@left and \pgf@circ@res@up}
}

\pgfcirc@declare@iecplug{iecplugL}{%
        % notice: center is on the left side
        \anchor{center}{\northwest\pgf@y=0pt}
        % put the node text above and to the right, ignore depth
        % the text is higher to match the iec socket position
        \anchor{text}{%
            \pgfextractx{\pgf@circ@res@left}{\northwest}
            \pgfextracty{\pgf@circ@res@up}{\northwest}
            \pgfpoint{-\wd\pgfnodeparttextbox+\pgf@circ@res@left}{
              .5\ht\pgfnodeparttextbox+2\pgf@circ@res@up
            }
        }
}
\pgfcirc@declare@iecplug{iecplugR}{%
        % notice: center is on the left side
        \anchor{center}{\northwest\pgf@y=0pt}
        % put the node text above and to the right, ignore depth
        % the text is higher to match the iec socket position
        \anchor{text}{%
            \pgfextractx{\pgf@circ@res@left}{\northwest}
            \pgfextracty{\pgf@circ@res@up}{\northwest}
            \pgfpoint{-\pgf@circ@res@left}{
              .5\ht\pgfnodeparttextbox+2\pgf@circ@res@up
            }
        }
}
% BNC connector

\pgfdeclareshape{bnc}{
    \savedmacro{\ctikzclass}{\edef\ctikzclass{connectors}}
    \saveddimen{\scaledRlen}{\pgfmathsetlength{\pgf@x}{\ctikzvalof{\ctikzclass/scale}\pgf@circ@Rlen}}
    \anchor{center}{
        \pgfpointorigin
    }
    % BNC size is 2.5 times the size of the internal "ocirc", when class scale is=1
    \savedanchor\northwest{%
        \pgfmathsetlength{\pgf@circ@scaled@Rlen}{\ctikzvalof{\ctikzclass/scale}\pgf@circ@Rlen}
        \pgf@y=\ctikzvalof{nodes width}\pgf@circ@scaled@Rlen
        \pgf@y=2.5\pgf@y
        \pgf@x=-\pgf@y
    }
    % center is on the opening
    \anchor{center}{\northwest\pgf@y=0pt\pgf@x=-\pgf@x}
    \anchor{zero}{\pgfpointorigin}
    \anchor{hot}{\northwest\pgf@y=0pt\pgf@x=-\pgf@x}
    \anchor{shield}{\northwest\pgf@x=0pt\pgf@y=-\pgf@y}
    % geo-anchors
    \pgfcirc@northwest@symmetric@geoanchors
    % put the node text above and centered
    \anchor{text}{\pgfextracty{\pgf@circ@res@up}{\northwest}
        \pgfpoint{-.5\wd\pgfnodeparttextbox}{
            \dimexpr.5\dp\pgfnodeparttextbox+.5\ht\pgfnodeparttextbox+\pgf@circ@res@up\relax
        }
    }
    \anchorborder{
        \pgf@circ@res@left=\pgf@x
        \pgf@circ@res@up=\pgf@y
        \pgfmathsetlength{\pgf@circ@scaled@Rlen}{\ctikzvalof{\ctikzclass/scale}\pgf@circ@Rlen}
        \pgfpointborderellipse{\pgfpoint{\pgf@circ@res@left}{\pgf@circ@res@up}
        }{\pgfpoint{2.5*\ctikzvalof{nodes width}*\pgf@circ@scaled@Rlen}{2.5*\ctikzvalof{nodes width}*\pgf@circ@scaled@Rlen}}
    }
    \pgf@circ@draw@component{
        \pgfextracty{\pgf@circ@res@other}{\northwest}
        \pgfmathsetlength{\pgf@circ@res@step}{\ctikzvalof{nodes width}*\scaledRlen}
        \pgfstartlinewidth=\pgflinewidth
        \pgf@circ@setlinewidth{bipoles}{\pgflinewidth}
        \pgf@circ@setcolor
        % external circle
        \pgfscope
            % clipping path: first a rectangle bigger then the shape
            % to avoid problems with the line thickness
            \pgfpathrectanglecorners{\pgfpoint{-2\pgf@circ@res@other}{-2\pgf@circ@res@other}}
                {\pgfpoint{2\pgf@circ@res@other}{2\pgf@circ@res@other}}
            % next the opening to the right
            \pgfpathrectanglecorners{\pgfpoint{-\pgf@circ@res@step}{-\pgf@circ@res@step}}
                {\pgfpoint{2\pgf@circ@res@other}{\pgf@circ@res@step}}
            % do the difference and clip before drawing
            \pgfseteorule
            \pgfusepath{clip}
            \pgfpathcircle{\pgfpointorigin}{\pgf@circ@res@other}
            \pgfusepath{draw}
        \endpgfscope
        % internal circle
        \pgfpathcircle{\pgfpointorigin}{\pgf@circ@res@step}
        \pgf@circ@draworfill
        % and the contact line to the right
        \pgfsetlinewidth{\pgfstartlinewidth}
        \pgfpathmoveto{\pgfpoint{\pgf@circ@res@step}{0pt}}
        \pgfpathlineto{\pgfpoint{\pgf@circ@res@other}{0pt}}
        \pgfusepath{draw}
    }
}
% %>>>

\endinput
% vim: set fdm=marker fmr=%<<<,%>>>:
