% Copyright 2019 by Till Tantau
%
% This file may be distributed and/or modified
%
% 1. under the LaTeX Project Public License and/or
% 2. under the GNU Public License.
%
% See the file doc/generic/pgf/licenses/LICENSE for more details.

\ProvidesFileRCS{pgfmoduleplot.code.tex}


% PGF's plotting interface works as follows:
%
% In order to plot something, two things need to be done. First, you
% need to provide the coordinates (obviously) of the points that
% should be plotted. The coordinates are given via a long stream of
% commands. These commands are \pgfplotstreamstart, which is
% given exactly once at the beginning, the commands
% \pgfplotstreampoint, \pgfplotstreampointoutlier,
% \pgfplotstreampointundefined, \pgfplotstreamnewdataset, of which
% there are numerous in the middle, the special \pgfplotstreamspecial,
% of which there may be numerous in the middle, and \pgfplotstreamend,
% which must be given at the end. Between these commands arbitrary
% other commands may be given. Here is an example:
%
% ...
% \pgfplotstreamstart
% \pgfplotstreampoint{\pgfpointxy{0}{0}}
% \pgfplotstreampoint{\pgfpointxy{1}{1}}
% \pgfplotstreamnewdataset
% \pgfplotstreampoint{\pgfpointxy{2}{4}}
% \relax
% \pgfplotstreampointoutlier{\pgfpointxy{3}{9}}
% \pgfplotstreamspecial{some handler-dependent special stuff}
% \pgfplotstreamend
%
% By themselves, the \pgfplotstreamxxxx commands do not do anything by
% default. Rather, to ``use'' such a stream, you must first install a
% stream handler. For example, the ``lineto'' handler will simply
% basically every \pgfplotstreampoint into a \pgfpathlineto.
%
% One special things is the handling of "jumps" in a stream. For
% instance, when a lineto handler encounters an "outlier" or a new
% data set, the current line should end and a new subpath should start
% (unless configured otherwise). For this, the special jump handler is
% important.
%
% Example:
%
% \pgfpathmoveto{\pgfpointorigin}
%
% \pgfplothandlerlineto
% \pgfplotstreamstart
% \pgfplotstreampoint{\pgfpointxy{0}{0}}
% \pgfplotstreampoint{\pgfpointxy{1}{1}}
% \pgfplotstreampoint{\pgfpointxy{2}{4}}
% \relax
% \pgfplotstreampoint{\pgfpointxy{3}{9}}
% \pgfplotstreamend


% The stream commands actually call their ``internal'' versions, which
% are set by the handlers:
\def\pgfplotstreamstart{\pgf@plotstreamstart}%
\def\pgfplotstreampoint#1{\gdef\pgfplotlastpoint{#1}\pgf@plotstreampoint{#1}}%
\def\pgfplotstreampointoutlier#1{\pgfplot@outliers{}{\pgf@plotstreamjump}{\pgfplotstreampoint{#1}}}%
\def\pgfplotstreampointundefined{\pgfplot@outliers{}{\pgf@plotstreamjump}{}}%
\def\pgfplotstreamnewdataset{\pgfplot@outliers{}{\pgf@plotstreamjump}{}}%
\def\pgfplotstreamspecial{\pgf@plotstreamspecial}%
\def\pgfplotstreamend{\pgf@plotstreamend}%

\def\pgfplot@ignorer#1#2#3{#1}%
\def\pgfplot@jumper#1#2#3{#2}%
\def\pgfplot@plotter#1#2#3{#3}%

\let\pgfplot@outliers\pgfplot@jumper
\let\pgfplot@undefined\pgfplot@jumper
\let\pgfplot@newdata\pgfplot@jumper

\pgfset{
  handle outlier points in plots/.is choice,
  handle outlier points in plots/ignore/.code=\let\pgfplot@outliers\pgfplot@ignorer,
  handle outlier points in plots/jump/.code=\let\pgfplot@outliers\pgfplot@jumper,
  handle outlier points in plots/plot/.code=\let\pgfplot@outliers\pgfplot@plotter,
  handle undefined points in plots/.is choice,
  handle undefined points in plots/ignore/.code=\let\pgfplot@undefined\pgfplot@ignorer,
  handle undefined points in plots/jump/.code=\let\pgfplot@undefined\pgfplot@jumper,
  handle new data sets in plots/.is choice,
  handle new data sets in plots/ignore/.code=\let\pgfplot@newdata\pgfplot@ignorer,
  handle new data sets in plots/jump/.code=\let\pgfplot@newdata\pgfplot@jumper,
}%


% Declares a new plot handler
%
% #1 = macro that should install the stream handler subsequently
% #2 = parameter list of said macro.
% #3 = keys as described below
%
% Description:
%
% When you declare a new plot handler, you provide, through keys,
% different "actions" that should be taken when the different stream
% commands are used:
%
% start       = some code to be executed when \pgfplotstreamstart is called
%               for this plothandler
% end         = same for the end of the stream
% point       = code to be executed for a point of the stream
% jump        = code to be executed for a "jump" in the stream
% special     = code to be executed for \pgfplotstreamspecial
% start macro   = here and for the following keys, the to-be-executed
%                 code is stored in the given macro
% end macro     = see above
% jump macro    = see above
% point macro   = see above
% special macro = see above
%
% Example:
%
% \pgfdeclareplothandler{\ultrasimplehandler}{}{
%   point=\pgfpathlineto{##1}
% }

\def\pgfdeclareplothandler#1#2#3{%
  \def#1#2{%
    \pgfkeys{%
      /pgf/plots/@handler options/.cd,
      start=\relax,
      end macro=\relax,
      point macro=\pgfutil@gobble,
      jump macro=\relax,
      special macro=\pgfutil@gobble,%
      #3%
    }%
  }%
}%

\pgfkeys{%
  /pgf/plots/@handler options/.cd,
  start/.code=%
    \gdef\pgf@plotstreamstart{%
      \global\pgf@plot@startedfalse%
      \global\let\pgf@plotstreamend\pgf@plotstreamend@init%
      \global\let\pgf@plotstreampoint\pgf@plotstreampoint@init%
      \global\let\pgf@plotstreamjump\pgf@plotstreamjump@init%
      \global\let\pgf@plotstreamspecial\pgf@plotstreamspecial@init%
      #1%
    },%
  start macro/.code=%
    \gdef\pgf@plotstreamstart{%
      \global\pgf@plot@startedfalse%
      \global\let\pgf@plotstreamend\pgf@plotstreamend@init%
      \global\let\pgf@plotstreampoint\pgf@plotstreampoint@init%
      \global\let\pgf@plotstreamjump\pgf@plotstreamjump@init%
      \global\let\pgf@plotstreamspecial\pgf@plotstreamspecial@init%
      #1%
    },%
  end/.code=\gdef\pgf@plotstreamend@init{#1},
  end macro/.code=\global\let\pgf@plotstreamend@init#1,
  point/.code=\gdef\pgf@plotstreampoint@init##1{#1},
  point macro/.code=\global\let\pgf@plotstreampoint@init#1,
  jump/.code=\gdef\pgf@plotstreamjump@init{#1},
  jump macro/.code=\global\let\pgf@plotstreamjump@init#1,
  special/.code=\gdef\pgf@plotstreamspecial@init##1{#1},
  special macro/.code=\global\let\pgf@plotstreamspecial@init#1,
}%

\newif\ifpgf@plot@started




% Sets the action taken for the first point of a plot to a lineto.
%
% Description:
%
% For certain handlers it makes sense either the start a plot by
% moving to the first point of the plot or to do a lineto to that
% first point. Using this command this action can be set to a lineto.
%
% Example:
%
% \pgfsetlinetofirstplotpoint

\def\pgfsetlinetofirstplotpoint{\let\pgf@plot@first@action=\pgfpathlineto}%


% Sets the action taken for the first point of a plot to a moveto.
%
% Example:
%
% \pgfsetmovetofirstplotpoint

\def\pgfsetmovetofirstplotpoint{\let\pgf@plot@first@action=\pgfpathmoveto}%

\let\pgf@plot@first@action=\pgfpathmoveto




%
% Handlers
%


% This handler converts each plot stream command into a lineto
% command, except for the first, which is converted to the action that
% has previously been specified using \pgfsetlinetofirstplotpoint or
% \pgfsetmovetofirstplotpoint.
%
% Example:
%
% \pgfplothandlerlineto
% \pgfplotxyfile{mytable}

\pgfdeclareplothandler{\pgfplothandlerlineto}{}{
  point macro=\pgf@plot@line@handler,
  jump=\global\let\pgf@plotstreampoint\pgf@plot@line@handler@move%
}%

\def\pgf@plot@line@handler#1{%
  \pgf@plot@first@action{#1}%
  \global\let\pgf@plotstreampoint=\pgfpathlineto%
}%

\def\pgf@plot@line@handler@move#1{%
  \pgfpathmoveto{#1}%
  \global\let\pgf@plotstreampoint=\pgfpathlineto%
}%



% This handler turns creates a series of lineto commands, with the
% last command being a closepath, resulting in a closed path. If a
% jump is encountered, the current subpath is closed and a new subpath
% is started.
%
% Example:
%
% \pgfplothandlerpolygon
% \pgfplotxyfile{mytable}

\pgfdeclareplothandler{\pgfplothandlerpolygon}{}{%
  point macro=\pgf@plot@line@handler@close,
  jump macro=\pgf@plot@next@close@and@moveto,
  end macro=\pgf@plot@polygon@stop
}%

\def\pgf@plot@line@handler@close#1{%
  \pgfpathmoveto{#1}%
  \global\pgf@plot@startedtrue%
  \global\let\pgf@plotstreampoint=\pgfpathlineto%
}%

\def\pgf@plot@polygon@stop{%
  \ifpgf@plot@started%
    \pgfpathclose%
  \fi%
  \global\pgf@plot@startedfalse%
}%

\def\pgf@plot@next@close@and@moveto{%
  \ifpgf@plot@started%
    \pgfpathclose%
  \fi%
  \global\let\pgf@plotstreampoint\pgf@plot@line@handler@close%
}%


% More handlers are defined in pgflibraryplothandlers





% This handler discards the plot.
%
% Example:
%
% \pgfplothandlerdiscard
% \pgfplotxyfile{mytable}

\pgfdeclareplothandler{\pgfplothandlerdiscard}{}{}%



% This handler records each plot stream command to a macro. This is
% useful if plot commands are difficult to generate and need to be
% ``recycled'' later on.
%
% Example:
%
% \pgfplothandlerrecord{\myplot}
% \pgfplotxyfile{mytable}  % stored in \myplot now
% \pgfplothandlerline
% \myplot
% \pgftransformxshift{1cm}
% \myplot

\pgfdeclareplothandler{\pgfplothandlerrecord}{#1}{%
  start=\gdef#1{\pgfplotstreamstart},
  point=\expandafter\gdef\expandafter#1\expandafter{#1\pgfplotstreampoint{##1}},
  jump=\expandafter\gdef\expandafter#1\expandafter{#1\pgf@plotstreamjump},
  special=\expandafter\gdef\expandafter#1\expandafter{#1\pgfplotstreamspecial{##1}},
  end=\expandafter\gdef\expandafter#1\expandafter{#1\pgfplotstreamend},
}%



% Read a plot stream from a file and plot it.
%
% #1 = file from which to read things
%
% File format:
%
% Each line of the file should begin with two numbers separated by a
% space. Such a line with number #1 and #2 is converted to a
% \pgfplotstreampoint{\pgfpointxy{#1}{#2}}. Extra characters following
% on the line are ignored.
%
% Lines starting with ``%'' and ``#'' are ignored.
%
% Example:
%
% \pgfplotxyfile{tableformgnuplot.dat}

\def\pgfplotxyfile#1{%
  \begingroup%
    \def\b@pgfplotsxyfile@scanning@for@first{1}%
    \pgfplotstreamstart%
    \openin\r@pgf@reada=#1
    \ifeof\r@pgf@reada
      \pgferror{Plot data file `#1' not found.}
     \else
      \catcode`\#=14
      \catcode`\^^M=5
      \pgf@readxyfile%
    \fi
    \pgfplotstreamend%
  \endgroup%
}%

\let\pgf@savedpar=\par
\def\pgf@partext{\par}%
\def\pgf@readxyfile{%
  \pgfutil@read\r@pgf@reada to \pgf@temp%
  \let\par=\pgf@savedpar%
  \edef\pgf@temp{\pgf@temp}%
  \ifx\pgf@temp\pgfutil@empty%
    \if1\b@pgfplotsxyfile@scanning@for@first
    \else
        \ifeof\r@pgf@reada\else\pgfplotstreamnewdataset\fi%
    \fi
  \else\ifx\pgf@temp\pgf@partext%
    \if1\b@pgfplotsxyfile@scanning@for@first
    \else
        \ifeof\r@pgf@reada\else\pgfplotstreamnewdataset\fi%
    \fi
  \else%
    \expandafter\pgf@parsexyline\pgf@temp\pgf@stop%
  \fi\fi%
  \ifeof\r@pgf@reada\else\expandafter\pgf@readxyfile\fi%
}%

\def\pgf@parsexyline#1 #2 #3\pgf@stop{%
  \def\b@pgfplotsxyfile@scanning@for@first{0}%
  \edef\pgf@xyline@flag@val{#3}%
  \ifx\pgf@xyline@flag@val\pgf@xyline@flag@undef%
    \pgfplotstreampointundefined%
  \else\ifx\pgf@xyline@flag@val\pgf@xyline@flag@out%
    \pgfplotstreampointoutlier{\pgfpointxy{#1}{#2}}%
  \else%
    \pgfplotstreampoint{\pgfpointxy{#1}{#2}}%
  \fi\fi%
}%

\edef\pgf@xyline@flag@out{o\space}%
\edef\pgf@xyline@flag@undef{u\space}%




% Read a plot stream from a file and plot it.
%
% #1 = file from which to read things
%
% File format:
%
% Like xy, except that each line contains three numbers, which are
% converted to xyz coordinates.
%
% Example:
%
% \pgfplotxyfile{tableformgnuplot.dat}

\def\pgfplotxyzfile#1{%
  \begingroup%
    \pgfplotstreamstart%
    \openin\r@pgf@reada=#1
    \ifeof\r@pgf@reada
      \pgferror{Plot data file `#1' not found.}
    \else
      \catcode`\#=14
      \catcode`\^^M=5
      \pgf@readxyzfile%
    \fi
    \pgfplotstreamend%
  \endgroup%
}%

\def\pgf@readxyzfile{%
  \pgfutil@read\r@pgf@reada to \pgf@temp%
  \ifx\pgf@temp\pgfutil@empty%
    \if1\b@pgfplotsxyfile@scanning@for@first
    \else
        \ifeof\r@pgf@reada\else\pgfplotstreamnewdataset\fi%
    \fi
  \else\ifx\pgf@temp\pgf@partext%
    \if1\b@pgfplotsxyfile@scanning@for@first
    \else
        \ifeof\r@pgf@reada\else\pgfplotstreamnewdataset\fi%
    \fi
  \else%
    \expandafter\pgf@parsexyzline\pgf@temp\pgf@stop%
  \fi\fi%
  \ifeof\r@pgf@reada\else\expandafter\pgf@readxyzfile\fi%
}%

\def\pgf@parsexyzline#1 #2 #3 #4\pgf@stop{%
  \def\b@pgfplotsxyfile@scanning@for@first{0}%
  \edef\pgf@xyline@flag@val{#4}%
  \ifx\pgf@xyline@flag@val\pgf@xyline@flag@undef%
    \pgfplotstreampointundefined%
  \else\ifx\pgf@xyline@flag@val\pgf@xyline@flag@out%
    \pgfplotstreampointoutlier{\pgfpointxyz{#1}{#2}{#3}}%
  \else%
    \pgfplotstreampoint{\pgfpointxyz{#1}{#2}{#3}}%
  \fi\fi%
}%




% Render a function using gnuplot.
%
% #1 = filename prefix for .gnuplot and .table files (optional,
%      default is \jobname)
% #2 = gnuplot function text
%
% Description:
%
% This command will write a file called #1.gnuplot that sets up
% some gnuplot commands to write output to a file called
% #1.table. Then it calls gnuplot (using the \write18 mechanism)
% to execute the file. Then it reads #2.table using \pgfplotxyfile.
%
% Example:
%
% \pgfplothandlerlineto
% \pgfplotgnuplot[\jobname]{plot [x=0:5] x*sin(x)}

{%
  \catcode`\%=12
  \catcode`\"=12
  \xdef\pgf@gnuplot@head{set table \noexpand\pgf@plottablefile@quoted; set format "%.5f"}
}%

\let\pgf@plotwrite=\w@pgf@writea
\newif\ifpgf@resample@plot

\pgfkeys{%
  /pgf/plot/gnuplot call/.initial={gnuplot}}%

\def\pgfplotgnuplot{\pgfutil@ifnextchar[{\pgf@plotgnuplot}{\pgf@plotgnuplot[\jobname]}}%}%
\def\pgf@plotgnuplot[#1]#2{%
  \pgf@resample@plottrue%
  \pgfutilpreparefilename{#1.gnuplot}%
  \let\pgf@plotgnuplotfile=\pgfretval
  \pgfutilpreparefilename{#1.table}%
  \let\pgf@plottablefile=\pgfretval
  \let\pgf@plottablefile@quoted=\pgfretvalquoted
  \edef\pgf@plot@code{\pgfutil@trimspaces@e{#2}}%
  \begingroup
  \endlinechar=-1 % suppress white space at end
  % Check, whether it is up-to-date
  \openin\pgfutil@inputcheck=\pgf@plotgnuplotfile\relax
  \ifeof\pgfutil@inputcheck%
  \else%
    \pgfutil@read\pgfutil@inputcheck to\pgf@temp% ignored
    \pgfutil@read\pgfutil@inputcheck to\pgf@plot@line%
    \closein\pgfutil@inputcheck
    \ifx\pgf@plot@code\pgf@plot@line%
      \openin\pgfutil@inputcheck=\pgfretval\relax
      \ifeof\pgfutil@inputcheck%
      \else%
        \closein\pgfutil@inputcheck
        \pgf@resample@plotfalse%
      \fi%
    \fi%
  \fi
  \ifpgf@resample@plot%
    \immediate\openout\pgf@plotwrite=\pgf@plotgnuplotfile\relax
    \immediate\pgfutil@write\pgf@plotwrite{\pgf@gnuplot@head}%
    \immediate\pgfutil@write\pgf@plotwrite{\pgf@plot@code}%
    \immediate\closeout\pgf@plotwrite%
    \pgfutil@shellescape{%
      \pgfkeysvalueof{/pgf/plot/gnuplot call} \pgf@plotgnuplotfile}%
  \fi%
  \endgroup
%  \let\pgf@savedparsexyline=\pgf@parsexyline%
%  \let\pgf@parsexyline=\pgf@parsegnuplotxyline%
  \pgfplotxyfile{\pgf@plottablefile}%
%  \let\pgf@parsexyline=\pgf@savedparsexyline%
}%

% \def\pgf@parsegnuplotxyline#1 #2 #3\pgf@stop{%
%   \edef\pgf@xyline@flag@val{#3}%
%   \edef\pgf@xyline@flag@out{o\space}%
%   \edef\pgf@xyline@flag@undef{u\space}%
%   \ifx\pgf@xyline@flag@val\pgf@xyline@flag@undef%
%     \pgfplotstreampointundefined%
%   \else\ifx\pgf@xyline@flag@val\pgf@xyline@flag@out%
%     \pgfplotstreampointoutlier{\pgfpointxy{#1}{#2}}%
%   \else%
%     \pgfplotstreampoint{\pgfpointxy{#1}{#2}}%
%   \fi\fi%
% }%

% This producer handler plots a function using pgf's mathematical engine.
%
% #1 = variable
% #2 = domain for the variable
% #3 = point, typically defined in terms of the value of the variable
%
% Description:
%
% This producer will iterate the variable #1 over all variables in #2
% (using the \foreach statement). For each value, a plot coordinate
% #3  is created.
%
% Note that this command is pretty slow.
%
% Example:
%
% \pgfplothandlerlineto
% \pgfplotfunction{\x}{0,0.1,...,3.141}{\pgfpointxy{\x}{sin(\x)}}

\def\pgfplotfunction#1#2#3{%
  \pgfplotstreamstart%
  \foreach#1in{#2}%
  {%
    \pgf@process{#3}%
    \edef\pgf@marshal{\noexpand\pgfplotstreampoint{\noexpand\pgfqpoint{\the\pgf@x}{\the\pgf@y}}}%
    \pgf@marshal%
  }
  \pgfplotstreamend%
}%




\endinput
