% This file is part of the glossaries bundle
% These are test glossary entries with URLs stored in
% the user1 field.

\newglossaryentry{aenean-url}{name={aenean},
 description={adipiscing auctor est},
 user1={http://uk.tug.org/}}

\newglossaryentry{morbi-url}{name={morbi},
 description={quam arcu, malesuada sed, volutpat et, elementum sit
amet, libero},
 user1={http://www.ctan.org/}}

\newglossaryentry{duis-url}{name={duis},
 description={mattis},
 user1={http://www.tug.org/}}

 % Now try an active character
\newglossaryentry{sed-url}{name={sed},
 description={cursus lectus quis odio (uses
\texttt{\string\protect\string~})},
 user1={http://theoval.cmp.uea.ac.uk/\protect~nlct/}}

\newglossaryentry{sed2-url}{name={sed},
 description={cursus lectus quis odio (uses
\texttt{\string\string\string~})},
 user1={http://theoval.cmp.uea.ac.uk/\string~nlct/}}

\newglossaryentry{sed3-url}{name={sed},
 description={cursus lectus quis odio (uses
\texttt{\string\glstildechar})},
 user1={http://theoval.cmp.uea.ac.uk/\glstildechar nlct/}}

 % How do we deal with a percent character?

 % Temporarily change the category code

\catcode`\%=12
\newglossaryentry{phasellus-url}{name={phasellus},
 description={arcu (catcode change)},
 user1=http://theoval.cmp.uea.ac.uk/%7Enlct
}
\catcode`\%=14

 % Use \%

\newglossaryentry{phasellus2-url}{name={phasellus},
 description={arcu (uses \texttt{\string\%})},
 user1=http://theoval.cmp.uea.ac.uk/\%7Enlct
}

 % Use \glspercentchar

\newglossaryentry{phasellus3-url}{name={phasellus},
 description={arcu  (uses
 \texttt{\string\glspercentchar})},
 user1=http://theoval.cmp.uea.ac.uk/\glspercentchar 7Enlct
}

